%%%%%%%%%%%%%%%%%%%%%%%%%%%%%%%%%%%%%%%%%
% Arsclassica Article
% LaTeX Template
% Version 1.1 (1/8/17)
%
% This template has been downloaded from:
% http://www.LaTeXTemplates.com
%
% Original author:
% Lorenzo Pantieri (http://www.lorenzopantieri.net) with extensive modifications by:
% Vel (vel@latextemplates.com)
%
% License:
% CC BY-NC-SA 3.0 (http://creativecommons.org/licenses/by-nc-sa/3.0/)
%
%%%%%%%%%%%%%%%%%%%%%%%%%%%%%%%%%%%%%%%%%

%----------------------------------------------------------------------------------------
%	PACKAGES AND OTHER DOCUMENT CONFIGURATIONS
%----------------------------------------------------------------------------------------

\documentclass[
10pt, % Main document font size
a4paper, % Paper type, use 'letterpaper' for US Letter paper
oneside, % One page layout (no page indentation)
%twoside, % Two page layout (page indentation for binding and different headers)
headinclude,footinclude, % Extra spacing for the header and footer
%BCOR5mm, % Binding correction
table
]{scrartcl}

%%%%%%%%%%%%%%%%%%%%%%%%%%%%%%%%%%%%%%%%%
% Arsclassica Article
% Structure Specification File
%
% This file has been downloaded from:
% http://www.LaTeXTemplates.com
%
% Original author:
% Lorenzo Pantieri (http://www.lorenzopantieri.net) with extensive modifications by:
% Vel (vel@latextemplates.com)
%
% License:
% CC BY-NC-SA 3.0 (http://creativecommons.org/licenses/by-nc-sa/3.0/)
%
%%%%%%%%%%%%%%%%%%%%%%%%%%%%%%%%%%%%%%%%%

%----------------------------------------------------------------------------------------
%	REQUIRED PACKAGES
%----------------------------------------------------------------------------------------

\usepackage[
nochapters, % Turn off chapters since this is an article        
beramono, % Use the Bera Mono font for monospaced text (\texttt)
eulermath,% Use the Euler font for mathematics
pdfspacing, % Makes use of pdftex’ letter spacing capabilities via the microtype package
dottedtoc % Dotted lines leading to the page numbers in the table of contents
]{classicthesis} % The layout is based on the Classic Thesis style



\usepackage[hmarginratio=1:1,top=25mm,left=20mm,columnsep=25pt]{geometry}
\usepackage{relsize} % e.g. used for \mathsmaller
\usepackage{bm}

\usepackage{arsclassica} % Modifies the Classic Thesis package

\usepackage[T1]{fontenc} % Use 8-bit encoding that has 256 glyphs

\usepackage[utf8]{inputenc} % Required for including letters with accents

\usepackage{graphicx} % Required for including images
\graphicspath{{Figures/}} % Set the default folder for images

\usepackage{enumitem} % Required for manipulating the whitespace between and within lists

\usepackage{lipsum} % Used for inserting dummy 'Lorem ipsum' text into the template

\usepackage{subfig} % Required for creating figures with multiple parts (subfigures)

\usepackage{amsmath,amssymb,amsthm} % For including math equations, theorems, symbols, etc

\usepackage{varioref} % More descriptive referencing

\usepackage{accents}

\usepackage[bottom]{footmisc}

\usepackage{titling} % it needs to define \thanksmarkseries

%----------------------------------------------------------------------------------------
% NEW COMMANDS
%----------------------------------------------------------------------------------------
\renewcommand{\AA}{\bm{A}}
\newcommand{\GG}{\bm{G}}
\newcommand{\MM}{\bm{M}}
\newcommand{\Id}{\bm{I}}
\newcommand{\vv}{\bm{v}}
\newcommand{\QQ}{\bm{Q}}
\renewcommand{\SS}{\bm{S}}
\newcommand{\Lie}{\mathfrak{L}}
\newcommand{\calE}{\mathcal{E}}						%

\newcommand{\IP}[1]{{\color{Red}IP:\ \ #1}}
\newcommand{\ER}[1]{{\color{Green}ER:\ \ #1}}
\newcommand{\BL}[1]{{\color{Cerulean}BL:\ \ #1}}
\newcommand{\NF}[1]{{\color{Plum}NF:\ \ #1}}


\newcommand{\sA}{\mathsmaller A}
\newcommand{\sB}{\mathsmaller B}
\newcommand{\sC}{\mathsmaller C}
\newcommand{\sD}{\mathsmaller D}
\newcommand{\sM}{\mathsmaller M}
\newcommand{\sN}{\mathsmaller N}
\newcommand{\sL}{\mathsmaller L}
\newcommand{\kronecker}[2]{\delta^{#1}_{\phantom{#1}#2}}
\newcommand{\durg}[2]{ D_{#1}^{\phantom{#1}#2} }	% Eulerian components of 
%the D field
\newcommand{\LeviCivitaUp}[1]{\varepsilon^{#1}}

\newcommand{\pd}{\partial}
\newcommand{\F}[2]{F^{\ #1}_{\mathsmaller#2}}
\newcommand{\hatF}[2]{\hat{F}^{\ #1}_{\mathsmaller#2}}
\newcommand{\A}[2]{A^{\mathsmaller#1}_{\ #2}}

\newcommand{\dist}[2]{ A^{#1}_{\phantom{#1}#2} }	% Component of the 
\newcommand{\Stress}[2]{ \Sigma^{#1}_{\phantom{#1}#2} }	% Total stress tensor
%distortion field A
\newcommand{\distsmall}[2]{ a_{{#1}{#2}} }	% Small distortion field A 
\newcommand{\Dist}{ \bm{A} }	% Distortion field A in matrix notations
\newcommand{\Burg}{ \bm{B} }	% Burgers tensor = curl(A)
\newcommand{\Durg}{ \bm{D} }	% Complimentary to the Burgers
\newcommand{\Distsmall}{ \bm{a} }	% Small distortion field A in matrix 
%notations
\newcommand{\Plastsmall}{ \bm{\pi} }	% Small distortion field A in matrix 
%notations
\newcommand{\Defgrad}{ \bm{F} }
\newcommand{\iDist}{ \bm{E} }
\newcommand{\symA}{\text{sym}(\bm{a})}
\newcommand{\skewA}{\text{skew}(\bm{a})}
\newcommand{\symP}{\text{sym}(\bm{\Plastsmall})}
\newcommand{\burg}[2]{ B^{{#1}{#2}} }	% Eulerian components of the B field
\newcommand{\itetr}[2]{e^{\phantom{#2}#1}_{#2}}
\newcommand{\tetr}[2]{a^{#1}_{\phantom{#1}#2}}
\newcommand{\rtetr}[2]{a^{#1}_{(\text{r}) #2}}
\newcommand{\spin}[2]{\omega^{#1}_{\phantom{#1}#2}}
\newcommand{\Lor}[2]{\Lambda^{#1'}_{\phantom{#1}#2}}
\newcommand{\iLor}[2]{\Lambda^{\phantom{#2}#1}_{#2'}}
\newcommand{\vel}[1]{v^{#1}}
\newcommand{\D}[1]{\mathcal{D}_{#1}} % Fock-Ivanencko cov derivative
\newcommand{\Tors}[2]{T^{#1}_{\phantom{a}#2}}
\newcommand{\Supp}[2]{S_{#1}^{\phantom{a}#2}}	%supepotential
\newcommand{\Torsl}[1]{T_{#1}}
\newcommand{\ET}[2]{E^{#1}_{\phantom{#1}#2}}	%Torsion decomposition, analog 
%of Electric field
\newcommand{\eT}[2]{D_{#1}^{\phantom{#1}#2}}	%Torsion decomposition, analog 
%of Electric field
\newcommand{\BT}[2]{B^{#1#2}}	%Torsion decomposition, analog of Magnetic field
\newcommand{\hT}[2]{H^{#1#2}}	%Torsion decomposition, analog of Magnetic field
\newcommand{\W}[2]{\mathcal{W}^{#1}_{\phantom{#1}#2}}
\newcommand{\w}[2]{W^{#1}_{\phantom{#1}#2}}
\newcommand{\FI}{Fock-Ivanenko}
\newcommand{\We}{Weitzenb\"ock}
\newcommand{\Lag}{\mathcal{L}}	% Lagrangian which depends on ordinary 
%derivatives
\newcommand{\Lagcov}{\pounds}% Lagrangian which depends on gauge covariant 
%derivatives
\newcommand{\Laghodge}{L}% Lagrangian which depends on the Hodge dual of the 
%torsion
\newcommand{\Lagtors}{\mathbb{L}}% Lagrangian which depends on torsion
\newcommand{\LagBE}{\mathfrak{L}}% Lagrangian which depends on the B and E 
%fields
\newcommand{\veps}{\varepsilon}
\newcommand{\EM}[2]{\Sigma^{#1}_{\phantom{#1}#2}}
\newcommand{\transpose}{{\mathrm {\mathsmaller T}}}
\newcommand{\tr}{\text{tr}}

\newcommand{\tegr}{TEGR}
\newcommand{\HT}[1]{\accentset{\star}{T}^{#1}}

\newcommand{\csh}{c_\text{sh}}	% 	shear sound speed
\newcommand{\csp}{c_\text{sp}}	% 	a sound speed related to the spin
\newcommand{\cinf}{c_\infty}	% 	a sound speed related to 
%1/\sqrt{\mu\eps}
%----------------------------------------------------------------------------------------
%	THEOREM STYLES
%---------------------------------------------------------------------------------------

\theoremstyle{definition} % Define theorem styles here based on the definition style (used for definitions and examples)
\newtheorem{definition}{Definition}

\theoremstyle{plain} % Define theorem styles here based on the plain style (used for theorems, lemmas, propositions)
\newtheorem{theorem}{Theorem}

\theoremstyle{remark} % Define theorem styles here based on the remark style (used for remarks and notes)

%----------------------------------------------------------------------------------------
%	HYPERLINKS
%---------------------------------------------------------------------------------------




%----------------------------------------------------------------------------------------
%	BIBLATEX
%---------------------------------------------------------------------------------------

\usepackage[backend=bibtex,giveninits=true,url=false,doi=true,eprint=true,isbn=false,
backref,backrefstyle=none,maxbibnames=99]{biblatex}
\DefineBibliographyStrings{english}{%
  backrefpage = {Cited on p\adddot},%
  backrefpages = {Cited on pp\adddot}%
}

\bibliography{library}

\renewcommand*{\bibfont}{\footnotesize}

% in order to suppress 'In:'
\renewbibmacro{in:}{%
  \ifboolexpr{%
     test {\ifentrytype{article}}%
  }{}{\printtext{\bibstring{in}\intitlepunct}}%
}

%----------------------------------------------------------------------------------------
% these commands allow to put equations in a fancy boxes:
%----------------------------------------------------------------------------------------
\usepackage{empheq}
\newlength\mytemplen
\newsavebox\mytempbox
\makeatletter
\definecolor{cream}{rgb}{.81, .88, 1}
 \newcommand\mycreambox{%
     \@ifnextchar[%]
        {\@mycreambox}%
        {\@mycreambox[0pt]}}
 \def\@mycreambox[#1]{%
     \@ifnextchar[%]
        {\@@mycreambox[#1]}%
        {\@@mycreambox[#1][0pt]}}
 \def\@@mycreambox[#1][#2]#3{
     \sbox\mytempbox{#3}%
     \mytemplen\ht\mytempbox
     \advance\mytemplen #1\relax
     \ht\mytempbox\mytemplen
     \mytemplen\dp\mytempbox
     \advance\mytemplen #2\relax
     \dp\mytempbox\mytemplen
     \colorbox{cream}{\hspace{1em}\usebox{\mytempbox}\hspace{1em}}}
 \makeatother

% ------------------------------------------------------------------------------
\newcommand*\samethanks[1][\value{footnote}]{\footnotemark[#1]}

 % Include the structure.tex file which specified the document structure and 
%layout

\PassOptionsToPackage{hyperfootnotes=true}{hyperref}
%[
%%draft, % Uncomment to remove all links (useful for printing in black and 
%%%white)
%colorlinks=true, 
%breaklinks=true, 
%bookmarks=true,
%bookmarksnumbered,
%urlcolor=webbrown, 
%linkcolor=RoyalBlue, 
%citecolor=webgreen, % Link colors
%pdftitle={}, % PDF title
%pdfauthor={\textcopyright}, % PDF Author
%pdfsubject={}, % PDF Subject
%pdfkeywords={}, % PDF Keywords
%pdfcreator={pdfLaTeX}, % PDF Creator
%pdfproducer={LaTeX with hyperref and ClassicThesis}, % PDF producer
%hyperfootnotes=true
%]

%----------------------------------------------------------------------------------------
%	TITLE AND AUTHOR(S)
%----------------------------------------------------------------------------------------

\title{\large\normalfont\spacedallcaps{Modeling phononic band gap in the 
Riemann-Cartan geometry framework}} % The article 
%title

%\subtitle{Subtitle} % Uncomment to display a subtitle

\author{
%	author1:
\normalsize\textsc{Lo\"ic Le Marrec}\thanks{Université de 
	Rennes I, IRMAR, Rennes, France},\qquad
%	author2:
\normalsize\textsc{Ilya Peshkov},\thanks{University of Trento, Trento, Italy}$\ \, ^, $\thanks{The 
work by I.P. has been started while being at Paul Sabatier 	University, IMT, Toulouse, France}\ \ 
%	author3:
%\normalsize\textsc{Van Hoi Nguyen(?)}\samethanks[1],\ \
%%	author4:
%\normalsize\textsc{Evgeniy Romenski(?)}\thanks{Sobolev 
%	Institute of Mathematics, Novosibirsk, Russia}$\ \, ^, $\thanks{Novosibirsk 
%	State University, Novosibirsk, Russia}
%\ldots 
}
\thanksmarkseries{arabic}
% The article author(s) - author afiliations 
%need to be 
%specified in the 
%AUTHOR AFFILIATIONS block

\date{\small\today} % An optional date to appear under the author(s)

%----------------------------------------------------------------------------------------


\begin{document}

%----------------------------------------------------------------------------------------
%	HEADERS
%----------------------------------------------------------------------------------------

\renewcommand{\sectionmark}[1]{\markright{\spacedlowsmallcaps{#1}}} % The header for all pages 
%(oneside) or for even pages (twoside)
%\renewcommand{\subsectionmark}[1]{\markright{\thesubsection~#1}} % Uncomment when using the 
%%twoside option - this modifies the header on odd pages
\lehead{\mbox{\llap{\small\thepage\kern1em\color{halfgray} 
\vline}\color{halfgray}\hspace{0.5em}\rightmark\hfil}} % The header style

\pagestyle{scrheadings} % Enable the headers specified in this block

%----------------------------------------------------------------------------------------
%	TABLE OF CONTENTS & LISTS OF FIGURES AND TABLES
%----------------------------------------------------------------------------------------

\maketitle % Print the title/author/date block

\setcounter{tocdepth}{2} % Set the depth of the table of contents to show sections and subsections 
%only

\tableofcontents % Print the table of contents

% \listoffigures % Print the list of figures

% \listoftables % Print the list of tables

%----------------------------------------------------------------------------------------
%	ABSTRACT
%----------------------------------------------------------------------------------------

\section*{Abstract} % This section will not appear in the table of contents due to the star 
% (\section*)
Modeling of acoustic waves in dispersive solids is discussed in the context of 
the Riemann-Cartan geometry. We propose a continuum model which is formulated 
in terms of the main objects of the Riemann-Cartan geometry such as the 
non-holonomic triad (distortion field) and torsion field. It is 
demonstrated that the model is able to describe the phononic band gap in 
dispersive solids. 

%----------------------------------------------------------------------------------------
%	AUTHOR AFFILIATIONS
%----------------------------------------------------------------------------------------
%\let\thefootnote\relax\footnotetext{* \textit{peshenator@gmail.com}}
%\let\thefootnote\relax\footnotetext{\textsuperscript{1} \textit{Paul Sabatier 
%University, IMT, Toulouse, France}}
%\let\thefootnote\relax\footnotetext{\textsuperscript{2} \textit{Université de 
%Rennes I, IRMAR, Rennes, France}}
%\let\thefootnote\relax\footnotetext{\textsuperscript{3} \textit{Sobolev 
%Institute of Mathematics, Novosibirsk, Russia}}
%\let\thefootnote\relax\footnotetext{\textsuperscript{4} \textit{Novosibirsk 
%State University, Novosibirsk, Russia}}
%\let\thefootnote\relax\footnotetext{\textsuperscript{4} \textit{Aix-Marseille 
%Université, IUSTI,  Marseille, France}}
%\let\thefootnote\relax\footnotetext{\textsuperscript{4} \textit{CNRS, LMA, 
%Marseille, France}}
%----------------------------------------------------------------------------------------

%\newpage % Start the article content on the second page, remove this if you have a longer abstract 
%that goes onto the second page

% PARAGRAPH OPTIONS:
\setlength\parindent{10pt} % sets indent to zero
\setlength{\parskip}{5pt} % changes vertical space between paragraphs
% PARAGRAPH OPTIONS.

%----------------------------------------------------------------------------------------
%	INTRODUCTION
%----------------------------------------------------------------------------------------

\section{Introduction}

\section{Governing Equations}

In the absence of irreversible effects due to plasticity or viscosity, the 
governing equations derived in \cite{PRD-Torsion2019} read 
\begin{subequations}\label{PDE}
	\begin{align}
	&\frac{\pd \rho}{\pd t} + \frac{\pd(\rho \vel{k})}{\pd x^k} = 
	0, \label{PDE.extend.rho}
	\\[2mm]
%
	&\frac{\pd M_i}{\pd t} + \frac{ \pd }{\pd x^k}  \left( M_i \vel{k} + P 
	\kronecker{k}{i} + 
	\dist{a}{i}
	\calE_{\dist{a}{k}} - \burg{a}{k} \calE_{\burg{a}{ i}} - \durg{a}{k} 
	\calE_{\durg{a}{ i}}
	\right ) = 0,\label{PDE.extend.M}
	\\[2mm]
%
	&\frac{\pd \dist{a}{k}}{\pd t} +\frac{\pd (  {\dist{a}{i} \vel{i}}  )}{\pd 
	x^k} + 
	\vel{j} \left(\frac{\pd \dist{a}{k}}{\pd x^j} - \frac{\pd\dist{a}{j}}{\pd 
	x^k}\right) = 
	-\frac{1}{\alpha} \calE_{\durg{a}{k}},\label{PDE.extend.A}
	\\[2mm]
%	
	&\frac{\pd \burg{a}{i}}{\pd t} + \frac{\pd}{\pd x^k} \left(
	\burg{a}{i} \vel{k} - \vel{i} \burg{a}{k} + \LeviCivitaUp{ikj} 
	\calE_{\durg{a}{j}}
	\right) + \vel{i} \frac{\pd \burg{a}{k}}{\pd x^k} = 0,\label{PDE.extend.B}
	\\[2mm]
%	
	&\frac{\pd \durg{a}{i}}{\pd t} + \frac{\pd} {\pd x^k} \left( \durg{a}{i} 
	\vel{k}  -  \vel{i} 
	\durg{a}{k} - \LeviCivitaUp{i k j}  \calE_{\burg{a}{j}} \right) + \vel{i} 
	\frac{\pd 
		\durg{a}{k}}{\pd x^k}  = \frac{1}{\alpha}\calE_{\dist{a}{i}}, 
		\label{PDE.extend.D}
	\\[2mm]
%	
	&\frac{\pd s}{\pd t} + \frac{\pd (s \vel{k})}{\pd x^k} = 0. 
	\label{PDE.extend.s}
	\end{align}
\end{subequations}

The following energy conservation law can be obtained as the consequence of the 
PDEs \eqref{PDE}
\begin{equation}\label{energy.law}
\frac{\pd \calE}{\pd t} + \frac{\pd }{\pd x_k} \left( \vel{k} \calE + \vel{i} 
\Stress{k}{i} + \LeviCivitaUp{ijk} 
\calE_{\durg{a}{i}}\calE_{\burg{a}{j}}\right) = 0,
\end{equation}
where $ \dist{a}{k} $ is 
the field of bases triads, non-holonomic in general, (also called the 
distortion field in 
\cite{PRD-Torsion2019,DPRZ2016}). The fields $ \burg{a}{i} $ and $ \durg{a}{i} 
$ can be viewed as two parts of the four-torsion $ \Tors{a}{\mu\nu} 
= \pd_\mu \dist{a}{\nu} - \pd_\nu \dist{a}{\mu} $, $ \mu,\nu=0,1,2,3 $, see 
details in 
\cite{PRD-Torsion2019}, so that $ \burg{a}{i} = \alpha
\LeviCivitaUp{ijk}\pd_j\dist{a}{k}$ contains only spatial derivatives of $ 
\dist{a}{k} $ with $ \alpha \sim L^{-1}$ being the scaling constant and $ L $ 
is a length unit, while $ \durg{a}{i} $ contains both time and space 
derivatives\footnote{Its 
explicit 
expression in terms of $ 
\pd_t 
\dist{a}{k} $ and $ \pd_i \dist{a}{k} $ is rather impossible because it 
involves Legendre 
transformations which are non-linear.}. The physical role of $ \burg{a}{i} $ is 
to represent the small scale structural incompatibility in the distortion 
field, while $ \durg{a}{i} $ represents the micro inertial effect of the 
microstructure.

Also, $ \rho $ is the mass density which is constrained by $ \rho = \rho_0 
\det(\dist{a}{k}) $ with $ \rho_0 $ being the reference mas density, $ \vel{i} 
$ is the velocity 
of the medium, $ 
M_i $ is the total momentum 
of the medium which includes not only the 
matter momentum but also it includes a contribution due to 
torsion. Its specification depends on the energy potential $ \calE $ and is
constraint by the relation $ \calE_{M_i} = \vel{i} $. Also, 
\begin{equation}\label{stress}
\Stress{k}{i} := -
P \kronecker{k}{i} - \dist{a}{i} \calE_{\dist{a}{k}} + 
\burg{a}{k} 
\calE_{\burg{a}{i}} + \durg{a}{k}\calE_{\durg{a}{i}}  
\end{equation}is the total stress 
tensor, $ P := \rho \calE_\rho + s\calE_s + M_i \calE_{M_i} + 
\burg{a}{i}\calE_{\burg{a}{i}} + 
\durg{a}{i}
\calE_{\durg{a}{i}} - \calE $ is the thermodynamic pressure, and the last term 
in the energy flux, $ \LeviCivitaUp{ijk} 
\calE_{\durg{a}{i}}\calE_{\burg{a}{j}} $, is the contribution due 
to torsion.

Because equation \eqref{PDE.extend.M} represents the conservation law for the 
total (matter + torsion) momentum, in order to satisfy the conservation law 
of the angular momentum, the total momentum flux $ M_i \vel{k} - \Stress{k}{i} 
$ has to be symmetric, see details in \cite{PRD-Torsion2019}. This, however, 
cannot be guarantied for arbitrary energy potential $ \calE $ and imposes 
certain constraints on the choice of $ \calE $. Thus, an example for $ \calE $ 
which provides the symmetric momentum flux was proposed in 
\cite{PRD-Torsion2019}
\begin{subequations}\label{EOS}
\begin{equation}
\calE = \rho \veps(\rho,s) + \rho \frac{\csh^2}{4} ||\GG'||^2 + 
\frac{1}{2\rho} \MM^2 +
\calE^t(\MM,\Burg,\Durg)
\end{equation}
where $ \bm{G} =\AA^\transpose\AA$, and $ \bm{G}' = \bm{G} -
\frac{\text{tr}(\bm{G})}{3}\bm{I} $ is the deviatoric part of $ \GG $, while
\begin{equation}\label{energy.torsion}
%{\red \frac12 \rho\, c_t^2 ||\skewA||^2} + 
\calE^t  = \frac{1}{2}\left (\frac1\epsilon \, 
||\Durg||^2 
+ 
\frac1\mu \, ||\Burg||^2\right ) 
%
-\frac{1}{\rho}\sum_{a=1}^{3}\left|
\begin{array}{ccc}
M_1 & \durg{a}{1} & \burg{a}{1} \\
M_2 & \durg{a}{2} & \burg{a}{2} \\
M_3 & \durg{a}{3} & \burg{a}{3}
\end{array}
\right|
%
- \csp\sum_{a=1}^{3}\left|
\begin{array}{ccc}
\dist{a}{1} & \durg{a}{2} & \burg{a}{1} \\
\dist{a}{2} & \durg{a}{2} & \burg{a}{2} \\
\dist{a}{3} & \durg{a}{3} & \burg{a}{3}
\end{array}
\right|
\end{equation}
\end{subequations}
Here, 
$ \epsilon $ and $ \mu $ are torsion related transport parameters 
which together scale as the inverse velocity square, $ (\epsilon\mu)^{-1}\sim 
v^2 $ so that we introduce the velocity
\begin{equation}\label{light.speed}
\cinf = \frac{1}{\sqrt{\epsilon\mu}}.
\end{equation}
As we shall see later, $ \cinf $ is the maximum velocity at which mechanical 
perturbations can propagate in the medium.

The parameter $ \csp $ has the dimension of velocity and characterizes 
the propagation of rotational degrees of freedom. Also, $ \csh $ is the sound 
speed of propagation of tangential perturbations. Both $ \csp $ and $ \csh $ 
are constant in this paper. Finally, there is a bulk 
sound speed $ c_0 $ associated with the hydrodynamic part of the equation of 
state, that is $ \varepsilon(\rho,s) $, and so that the longitudinal sound 
speed $ c_\text{l} $ is expressed conventionally as
\begin{equation}\label{long.sound}
c_\text{l}^2 = c_0^2 + \frac43 \csh^2. 
\end{equation}
Therefore, we distinguish the following  characteristic velocities
\begin{equation}\label{velocity.all}
c_0, \qquad \csh, \qquad c_\text{l} = \sqrt{c_0^2 + \frac43 \csh^2}, 
\qquad \csp, \qquad \cinf = \frac{1}{\sqrt{\epsilon\mu}}.
\end{equation}

Note that the equation of state \eqref{EOS} gives the following expression for 
the total momentum 
\begin{equation}\label{momentum.total}
\MM =\rho \vv + \Durg_a\times\Burg^a.
\end{equation}

In the rest of the paper, the tensorial character of the state variables $ \dist{a}{i} $, $ 
\durg{a}{i} $, and $ \burg{a}{i} $ will be not important and hence, for simplicity, we shall not 
distinguish between the Lagrangian indices $ a,b,c $ and Eulerian ones $ i,j,k $ but will use only 
$ 
i,j,k, \ldots $. Moreover, we shall write all indices down, e.g. 
\begin{equation}\label{notations}
\dist{a}{i} \to A_{ji}, \qquad \durg{a}{i} \to D_{ji}, \qquad \burg{a}{i} \to B_{ji}.
\end{equation}



\section{Small amplitude waves}

In 
the rest of the paper, we restrict our consideration to the one-dimensional case along the 
coordinate $ x:=x_1 $. Moreover, because we are interested in modeling of propagation of small 
amplitude waves, it is convenient to introduce the following state variables
\begin{equation}\label{key}
a_{ij} := \delta_{ij} - A_{ij}, \qquad d_{ij} := D_{ij}, \qquad b_{ij} := B_{ij},
\end{equation}
while the entropy $ s $ can be omitted from the consideration. 

Let us group the unknowns degrees of freedom in a single vector $q(t,x)$
\begin{equation}
\begin{array}{rcl}    
q(t,x)&=&\Big(\rho,v_{1},v_{2},v_{3}, \\
&&	\quad\quad	a_{11}, a_{21}, a_{31},
a_{12}, a_{22}, a_{32},
a_{13}, a_{23}, a_{33},\\
&& 	\quad\quad\quad 	b_{11}, b_{21}, b_{31},
b_{12}, b_{22}, b_{32},
b_{13}, b_{23}, b_{33},\\
&&  	\quad\quad\quad\quad d_{11}, d_{21}, d_{31},
d_{12}, d_{22}, d_{32},
d_{13}, d_{23}, d_{33}\Big)^T,
\end{array}
\end{equation}
where $(t,x)$ dependence has been omitted. 

By linearizing the non-linear system \eqref{PDE} near the equilibrium state
\begin{equation}\label{equilibrium.state}
\rho = \rho_0, \quad v_i = 0, \quad a_{ij} = \delta_{ij}, \quad d_{ij } = 0, \quad b_{ij} = 0,
\end{equation}
it can be written as a linear system
\begin{equation}\label{LinSyst}
\pd_{t}q+\mathbb{A}\pd_{x}q-\mathbb{B}q = 0
\end{equation}
where the matrices $\mathbb{A}$ and $\mathbb{B}$ are $31\times 31$ constant matrices which depend on
$\rho_{0}$, $c_{0}$, $\csh$, $\csp$, $\mu$, $\epsilon$, and $\alpha$ and are given in Appendix\,??

In this formulation, components of $q$ and entries of the matrices $\mathbb{A}$ and $\mathbb{B}$ 
do not have the same 
dimensions, and a clear analysis can be provided if one uses a fully dimensionless formulation, 
which is proposed in the following section.


\subsection{Nondimensionalization}

In this section, we consider a non-dimensional version of linear system \eqref{LinSyst}, from 
which, in particular, we shall see that there are only 3 non-dimensional governing parameters out 
of 5 velocities 
in \eqref{velocity.all}. 
Let us first introduce the diagonal matrice:
\begin{equation}
\mathbb{O}=\text{diag}\Big(\rho_{0}\quad,
	\underbrace{c_{0}}_\text{$3$ times },
	\underbrace{1}_\text{ $9$ times}, \quad
	\underbrace{c_{\infty}\sqrt{\mu\rho_{0}}}_{\text{$9$ times}}\quad ,\quad
	\underbrace{c_{0}\sqrt{\varepsilon\rho_{0}}}_{\text{$9$ times}} \Big)
\end{equation}
where the underbrace specifies the number of duplication of each term. Because the physical 
dimensions of the state variables are \cite{PRD-Torsion2019} ($ M $ is a unit of mass, $ L $ is 
a unit of length, $ T $ is a unit of time)
\begin{equation}
	\begin{array}{c}    
		\left[\rho\right]=\left[\rho_{0}\right]=\dfrac{M}{L^3}, \quad
		\left[v_{i}\right]=\left[c_{0}\right]=\dfrac{L}{T}, \quad
		\left[a_{ij}\right]=1,\\[2mm]
		\left[b_{ij}\right]=\left[c_{\infty}\sqrt{\mu\rho_{0}}\right]=\dfrac{1}{L^2}, \quad
		\left[d_{ij}\right]=\left[c_{0}\sqrt{\varepsilon\rho_{0}} \right]=\dfrac{M}{T}
	\end{array}
\end{equation}
we can define a non-dimensional vector of unknowns $\tilde{q}$ such that
\begin{equation}
q=\mathbb{O}\tilde{q}
\end{equation}
More precisely,
\begin{equation}\label{key}
\rho=\rho_{0}\, \tilde{\rho}, \quad
v_{i}=c_{0}\, \tilde{v}_{i}, \quad
a_{ij}=\tilde{a}_{ij}, \quad
b_{ij}=c_{\infty}\sqrt{\mu\rho_{0}}\, \tilde{b}_{ij}, \quad
d_{ij}=c_{0}\sqrt{\varepsilon\rho_{0}}\, \tilde{d}_{ij}
\end{equation}
Then, linear system (\ref{LinSyst}) can be written as 
\begin{equation}
\pd_{t}\tilde{q}+\mathbb{O}^{-1}\mathbb{A}\mathbb{O}\pd_{x}\tilde{q}-\mathbb{O}^{-1}\mathbb{B}\mathbb{O}\tilde{q}=0
\end{equation}
Here, all component of $\mathbb{O}^{-1}\mathbb{A}\mathbb{O}$ have the dimension of velocity whereas 
all components of  $\mathbb{O}^{-1}\mathbb{A}\mathbb{O}$ have the dimension of $T^{-1}$ (inverse 
time). At this moment, $t$ and $x$ are still dimensional. In order to introduce a pure non 
dimensional 
formulation let us define a characteristic length and time 
\begin{equation}\label{CharacteristicSize}
\ell=\alpha  \sqrt{\frac{\varepsilon}{\rho_{0}}}, \quad\quad 
\tau=\frac{\ell}{c_{0}}=\frac{\alpha}{c_{0}}  \sqrt{\frac{\varepsilon}{\rho_{0}}}.
\end{equation}
In such a way, the non-dimensional space $ \tilde{x} $ and time $ \tilde{t} $ are
\begin{equation}\label{Rescaling}
\tilde{t}=\frac{t}{\tau}, \qquad \tilde{x}=\frac{x}{\ell},
\end{equation}
while the partial derivatives transform as $\pd_{x}=\frac{1}{\ell}\pd_{\tilde{x}}$ and 
$\pd_{t}=\frac{1}{\tau}\pd_{\tilde{t}}$. 

Finally, we obtain the following non-dimensional linear 
system 
%\begin{equation}
%\frac{1}{\tau}\pd_{\tilde{t}}\tilde{q}+\mathbb{O}^{-1}\mathbb{A}\mathbb{O}\frac{1}{\ell}\pd_{\tilde{x}}\tilde{q}-\mathbb{O}^{-1}\mathbb{B}\mathbb{O}\tilde{q}=0
%\end{equation}
%\begin{equation}
%\pd_{\tilde{t}}\tilde{q}+\frac{\tau}{\ell}\mathbb{O}^{-1}\mathbb{A}\mathbb{O}\pd_{\tilde{x}}\tilde{q}-\tau
% \mathbb{O}^{-1}\mathbb{B}\mathbb{O}\tilde{q}=0
%\end{equation}
\begin{equation}
\pd_{\tilde{t}}\tilde{q}+\frac{1}{c_{0}}\mathbb{O}^{-1}\mathbb{A}\mathbb{O}\pd_{\tilde{x}}\tilde{q}-\tau
 \mathbb{O}^{-1}\mathbb{B}\mathbb{O}\tilde{q}=0
\end{equation}
Introducing matrices
\begin{equation}\label{nondim.matrices}
\tilde{\mathbb{A}} = \frac{1}{c_{0}}\mathbb{O}^{-1}\mathbb{A}\mathbb{O}
\quad\quad\quad
\tilde{\mathbb{B}} = \tau\,\mathbb{O}^{-1}\mathbb{B}\mathbb{O}
\end{equation}
we have the system of non-dimensional differential equations:
\begin{equation}\label{SystDiffAdim}
\boxed{\pd_{\tilde{t}}\tilde{q}+\tilde{\mathbb{A}}\pd_{\tilde{x}}\tilde{q}-\tilde{\mathbb{B}}\tilde{q}=0}
\end{equation}

\begin{remark} The matrices $\tilde{\mathbb{A}}$ and $\tilde{\mathbb{B}}$ are non-dimensional. 
However, they depend on the physical parameters through a limited set of non dimensional 
parameters. More precisely, introducing
\begin{equation}\label{CHIhpinf}
\chi_{h}=\frac{\csh}{c_{0}}, \quad\quad
\chi_{p}=\frac{\csp}{c_{0}}, \quad\quad
\chi_{\infty}=\frac{c_{\infty}}{c_{0}}
\end{equation}
we have explicitly $\tilde{\mathbb{A}}=\tilde{\mathbb{A}}(\chi_{h},\chi_{p},\chi_{\infty})$ and 
$\tilde{\mathbb{B}}=\tilde{\mathbb{B}}(\chi_{h},\chi_{p})$. In other word, the non-dimensional 
differential system (\ref{SystDiffAdim}) is completely controlled only by three independent 
parameters listed in (\ref{CHIhpinf}). The influence of others parameters ($\alpha$, $ \epsilon $, 
and $ \mu $) 
intervenes through space and/or time scaling factors according to (\ref{CharacteristicSize}) and 
(\ref{Rescaling}).
\end{remark} 

\begin{remark} The relation $\textbf{B}=\alpha \nabla\times \textbf{A}$ is reduced in 
this one-dimensional case to 
$$
b_{ij}=\alpha \varepsilon_{j1k}\pd_{x}a_{ik},
$$
and hence, according to  (\ref{CharacteristicSize}) and (\ref{Rescaling}) it becomes for the new 
non-dimensional variables
%$$
%c_{\infty}\sqrt{\mu\rho_{0}}\, 
%\tilde{b}_{ai}=\sqrt{\frac{\rho_{0}}{\varepsilon}}\varepsilon_{i1k}\pd_{\tilde{x}}\tilde{a}_{ak}
%$$
$$
\tilde{b}_{ij}=\varepsilon_{j1k}\pd_{\tilde{x}}\tilde{a}_{ik}
$$
or simply,
\begin{equation}\label{ABrelationAdim}
\tilde{b}_{i1}=0 , \quad \quad
\tilde{b}_{i2}=
%\varepsilon_{21k}\pd_{\tilde{x}}\tilde{a}_{ak}= \varepsilon_{213}\pd_{\tilde{x}}\tilde{a}_{a3}=
 -\pd_{\tilde{x}}\tilde{a}_{i3},
\quad\quad\quad
\tilde{b}_{i3}=
%\varepsilon_{31k}\pd_{\tilde{x}}\tilde{a}_{ak}= \varepsilon_{212}\pd_{\tilde{x}}\tilde{a}_{a2}=
 \pd_{\tilde{x}}\tilde{a}_{i2},
 \quad\quad
 \forall i=\{1,2,3\}.
\end{equation}
\end{remark}


\begin{remark} Expanding the differential system (\ref{SystDiffAdim}) we obtain:\\
$$
\begin{array}{rcl}
\pd_{t}{\tilde{\rho }}+\pd_{x}{\tilde{v}_1}&=&0,\\[1mm]
%MOMENTUM:
\pd_{t}{\tilde{v}_1} + \pd_{x}{\tilde{\rho
}} + \frac{2}{3} \chi _h^2 \pd_{x} \left(2   
\tilde{a}_{11} - \tilde{a}_{22} - \tilde{a}_{33}\right)&=&0,\\[1mm]
%
\pd_t\tilde{v}_2 + \chi _h^2   
\pd_{x} \left(\tilde{a}_{12} + \tilde{a}_{21}\right)&=&0,\\[1mm]
%
\pd_{t}\tilde{v}_3 + \chi_h^2 \pd_{x} \left(\tilde{a}_{13} + \tilde{a}_{31} \right)&=&0,\\[1mm]
%DISTORTION:
\pd_{t}{\tilde{a}_{11}}+\pd_{x}{\tilde{v}_1}&=&-\tilde{d}_{11},\\[1mm]
%    
\pd_{t}{\tilde{a}_{21}}+\pd_{x}{\tilde{v}_2}&=&- \tilde{d}_{21} - \chi _p\tilde{b}_{23},\\[1mm]
%
\pd_{t}{\tilde{a}_{31}}+\pd_{x}{\tilde{v}_3}&=& - \tilde{d}_{31} + \chi _p\tilde{b}_{32},\\[1mm]
%
\pd_{t}{\tilde{a}_{12}}&=&- \tilde{d}_{12} + \chi_p\tilde{b}_{13},\\[1mm]
%
\pd_{t}{\tilde{a}_{22}}&=&- \tilde{d}_{22},\\[1mm]
%
\pd_{t}{\tilde{a}_{32}}&=&-\tilde{d}_{32} - \chi _p\tilde{b}_{31},\\[1mm]
%
\pd_{t}{\tilde{a}_{13}}&=& - \tilde{d}_{13} -\chi_p\tilde{b}_{12},\\[1mm]
%
\pd_{t}{\tilde{a}_{23}}&=& -\tilde{d}_{23} + \chi_p\tilde{b}_{21},\\[1mm]%
%
\pd_{t}{\tilde{a}_{33}}&=& - \tilde{d}_{33},\\[1mm]
%BUrgers
\pd_{t}{\tilde{b}_{i1}}&=&0,\\[1mm]
%
\pd_{t}{\tilde{b}_{12}}- \pd_{x} (\chi_p{\tilde{b}_{12}} + {\tilde{d}_{13}})&=&0,\\[1mm]
%
\pd_{t}{\tilde{b}_{22}} + \pd_{x} (\chi _p{\tilde{b}_{21}} - {\tilde{d}_{23}})&=&0,\\[1mm]
%
\pd_{t}{\tilde{b}_{32}}-\pd_{x}{\tilde{d}_{33}}&=&0,\\[1mm]
%
\pd_{t}{\tilde{b}_{13}} - \pd_{x}(\chi_p\tilde{b}_{13} - \tilde{d}_{12})&=&0,\\[1mm]
%
\pd_{t}{\tilde{b}_{23}}+\pd_{x}{\tilde{d}_{22}}&=&0,\\[1mm]
%
\pd_{t}{\tilde{b}_{33}} + \pd_{x}(\chi_p\tilde{b}_{31}+\tilde{d}_{32})&=&0,\\[1mm]
%
\pd_{t}{\tilde{d}_{11}} &=& -  \frac{2}{3}\chi_h^2 
\left(-2\tilde{a}_{11}+\tilde{a}_{22}+\tilde{a}_{33}\right) ,\\[1mm]
%
\pd_{t}{\tilde{d}_{21}} &=& \chi _h^2\left(\tilde{a}_{12}+\tilde{a}_{21}\right) ,\\[1mm]
%
\pd_{t}{\tilde{d}_{31}} &=& \chi _h^2\left(\tilde{a}_{13}+\tilde{a}_{31}\right),\\[1mm]
%
\pd_{t}{\tilde{d}_{12}} +\chi _{\infty 
}^2\pd_{x}{\tilde{b}_{13}}-\chi _p\pd_{x}{\tilde{d}_{12}}&=& \chi _h^2 
\left(\tilde{a}_{12}+\tilde{a}_{21}\right),\\[1mm]
%
\pd_{t}{\tilde{d}_{22}} + \chi _{\infty 
}^2\pd_{x}{\tilde{b}_{23}}+\chi _p\pd_{x}{\tilde{d}_{21}}&=&-\frac{2}{3} \chi _h^2 
\left(\tilde{a}_{11}-2\tilde{a}_{22}+\tilde{a}_{33}\right),\\[1mm]
%
\pd_{t}{\tilde{d}_{32}} + \chi 
_{\infty}^2\pd_{x}{\tilde{b}_{33}}&=& \chi _h^2\left(\tilde{a}_{23}+\tilde{a}_{32}\right),\\[1mm]
%
\pd_{t}{\tilde{d}_{13}} -\chi _{\infty }^2 
\pd_{x}{\tilde{b}_{12}}-\chi _p\pd_{x}{\tilde{d}_{13}}&=& 
\chi_h^2\left(\tilde{a}_{13}+\tilde{a}_{31}\right),\\[1mm]
%
\pd_{t}{\tilde{d}_{23}} -\chi _{\infty}^2 
\pd_{x}{\tilde{b}_{22}}&=& \chi _h^2\left(\tilde{a}_{23}+\tilde{a}_{32}\right),\\[1mm]
%
\pd_{t}{\tilde{d}_{33}}  -\chi _{\infty 
}^2\pd_{x}{\tilde{b}_{32}}+\chi _p \pd_{x}{\tilde{d}_{31}}&=& -\frac{2}{3}\chi 
_h^2\left(\tilde{a}_{11}+\tilde{a}_{22}-2 \tilde{a}_{33}\right).
   \end{array}
   $$
\end{remark}
   
   
%-------------------------------------
\section{Dispersion relation}
%-------------------------------------
Let consider harmonic variations in time an space such that 
$$
\tilde{q}(t,x)=\tilde{Q}e^{i(\omega \tilde{t}-k\tilde{x})}
$$
then $\pd_{\tilde{x}}\tilde{q}(t,x)=-ik\tilde{q}(t,x)$ and 
$\pd_{\tilde{t}}\tilde{q}(t,x)=i\omega\tilde{q}(t,x)$, the differential system 
\eqref{SystDiffAdim} becomes the following algebraic ones
\begin{equation}
\left(\omega\mathbb{I} -k\tilde{\mathbb{A}}+i\tilde{\mathbb{B}}\right)Q=0
\end{equation}
in other words $Q$ is an eigenvector associated to the eigenvalue $\omega$ of the linear system $k\tilde{\mathbb{A}}-i\tilde{\mathbb{B}}$, \textit{ie}:
$$
\left(k\tilde{\mathbb{A}}-i\tilde{\mathbb{B}}\right)Q=\omega \, Q
$$
In practice this eigenvalues are the roots of the characteristic polynomial $\mathcal{D}(\omega,k)=\textrm{det}(\omega\mathbb{I} -k\tilde{\mathbb{A}}+i\tilde{\mathbb{B}})$. This dispersion equation may be factorized as follow
$$
\mathcal{D}(\omega,k)=\mathcal{D}_{0}\mathcal{D}_{1}\mathcal{D}_{2}\mathcal{D}_{3}\mathcal{D}_{4}
$$
where 
$$
\mathcal{D}_{0}=\omega^{11}, \quad\quad
\mathcal{D}_{1}=\omega^2-\chi_{\infty}^2k^2, \quad\quad
\mathcal{D}_{2}=\omega^2-2 \chi_{h}^2-\chi_{\infty}^2k^2
$$
and $\mathcal{D}_{3}$ is non 6 order order polynomial in terms of both $k$ and $\omega$, last $\mathcal{D}_{4}$  is 8 order order polynomial in terms of both $k$ and10 in terms of $\omega$. These two last polynomial are parametrized by $\chi_{h}$, $\chi_{h}$ and $\chi_{\infty}$. Their expressions are tricky and are not presented here. \\
\\
The dispersion equation $\mathcal{D}(\omega,k)=0$ can be solved in various way. For example, one 
can 
fix frequency $\omega$ and look for a wavenumber $k(\omega)$ for witch $\mathcal{D}(\omega,k)=0$ 
is satisfied. Reciprocally, we can fix $k$ and looks for $\omega(k)$ satisfying the dispersion 
equation. In the following we consider an arbitrary positive frequency and looks for some roots 
$k(\omega)\in \mathbb{C}$ of the dispersion equation.  The relation $k(\omega)$ is called in the 
following dispersion relation. At a given frequency a dispersion relation is associated to each 
eigenvector $Q$. The corresponding $Q$ is commonly called a mode. \\

\begin{remark}
The rescaling $\tilde{x}=x/\ell$ and $\tilde{t}=t/\tau$ affect the 
eigenfrequencies and wavenumber as well. Let us also introduce physical wavenumber $\underline{k}$ 
and physical frequency $\underline{\omega}$. We have of course 
\begin{equation}\label{phys.k.omega}
\underline{k}=\frac{k}{\ell}, \quad\quad\quad
\underline{\omega}=\frac{\omega}{\tau}
\end{equation}
\end{remark}


\begin{remark} According to the previous remark, if the eigenvector is associated to a zero of an 
homogenous polynomial in term of both $\omega$ and $k$, then  the rescaling do not affect the 
dispersion relation. At this stage only $\mathcal{D}_{0}$ and $\mathcal{D}_{1}$ have this property. 
\end{remark}

%-------------------------------------
\subsection{Critical values}
%-------------------------------------
Before analysing in details the dispersion relation, we consider some critical values of $\omega$ 
and $k$ associated to the dispersion relation.
%-------------------------------------
\subsubsection{Cut-off frequency}
%-------------------------------------
A cut-off frequency $\omega_{c}$ is a non-zero frequency associated to a zero-wavenumber, then a roots of $\mathcal{D}(\omega,0)$. At such frequency, some time oscillation appears but the variable are not space-harmonic. As $\mathcal{D}(\omega,0)=\omega^{21}(\omega^2-2 \chi_{h}^2)^5$ the only cut-off frequency is 
$$
\boxed{\omega_{c}=\chi_{h}\sqrt{2}}
$$
and is present for only 10 modes (considering $\pm$ solutions). Recovering the physical dimension, 
we have
 $$
\underline{\omega}_{c}=\frac{c_{sh}}{\alpha} \sqrt{\frac{2\rho_{0}}{\varepsilon}}
 $$
 hence this cut-off frequency is independent on $c_{0}$ and is not defined $\alpha=0$ or $\alpha\to\infty$. This cut-off frequency is observed for the modes associated to $\mathcal{D}_{2}$, $\mathcal{D}_{3}$ and $\mathcal{D}_{4}$.
 
 %-------------------------------------
\subsection{Breathing mode}
%-------------------------------------
A breathing mode is a space harmonic mode associated to a zero-frequency. Hence its wavenumber $k_{b}$ is solution of $\mathcal{D}(0,k_{b})=0$. 8 solutions may be obained, satisfying one of the following relations :
$$
\begin{array}{c}
k^2_{b}=-2\frac{\chi_{h}^2}{\chi_{\infty}^2}:=k^2_{b1}, \quad\quad
k^2_{b}=\frac{\chi_{p}^2}{\chi_{\infty}^2}-1:=k^2_{b2}, \\[8pt]
k^2_{b}=\frac{1}{2}\left(k^2_{b1}+k^2_{b2}\pm \sqrt{(k^2_{b1}+k^2_{b2})^2+4k^2_{b1}}\right):=k^2_{b3}
\end{array}
$$
It is completely controlled by the ratio $\frac{c_{sh}}{c_{\infty}}$ and 
$\frac{c_{p}}{c_{\infty}}$\\
These wavenumbers are either real (space-oscillation) , purely imaginary (exponential 
decay/increase in space) or complex. In a infinite domain complex wavenumber have to be proscribed 
in order to avoid space explosion toward infinity. However such type of mode are physically 
admissible in bounded domain where they may be observed in boundary layers.  \\
In terms of mode family, $k^2_{b1}$ is associated to $\mathcal{D}_{2}$, $k^2_{b2}$ is associated to $\mathcal{D}_{4}$, and $k^2_{b3}$ is associated to $\mathcal{D}_{3}$, \\
As for cut-off frequency, this breathing wavenumber is physically justified only if $\alpha$ is non-null but finite. 

 %-------------------------------------
\subsection{Zero group velocity}
%-------------------------------------
At a given frequency $\omega$, the group velocity of a specific mode is null if 
$$
\frac{\pd \omega}{\pd k}=0
$$
where $k(\omega)$ is the dispersion relation of the corresponding mode. (Rudin, W. (2006). Real and complex analysis. Tata McGraw-hill education.). In practice this phenomena occurs if the polynomial in $k$ associated to the dispersion relation, degenerates to a lower order. Hence if the monome associated to the higher power in $k$ is null. This occurs only for the dispersion relation $\mathcal{D}_{4}$ and for a specific frequency $\omega_{z}$ only: 
$$
\boxed{\omega_{z}=\pm\frac{2}{\sqrt{3}}\frac{\chi_{h}}{\sqrt {1+\frac{4}{3}\chi_{h}^2}}\quad =\pm\frac{2}{\sqrt{3}}\frac{c_{h}}{c_{l}}}
$$


%-------------------------------------
%<<<<<<< HEAD
%\section{Time-invariant  eigen-solution}
%%-------------------------------------
%Eleven solutions of \eqref{SystDiffAdim} are associated to time-invariant solution as $\omega=0$ 
%is 
%a multiple root of order $11$ according to $\mathcal{D}_{0}$. These solutions are analyzed below. 
%These solutions may seen as the eigenvectors of $k\tilde{\mathbb{A}}-i\tilde{\mathbb{B}}$ 
%associated a zero eigenvalues. The set of eigenvectors, may be decomposed by introducing sub-set 
%for which the component are independent. 
%%-------------------------------------
%\subsection{Trivial eigenvectors}
%%-------------------------------------
%We present here the eigenvectors containing only two non-null components. They are all independent 
%and no space derivative is involved. In other word they imply simple relation between degree of 
%freedom. 
%\begin{itemize}
%\item Two eigenvectors are related to the non-diagonal distorsion containing the propagation 
%direction. They impose skew-symmetry of these components:
%=======
\section{Stationnary problem}
%-------------------------------------
For a stationary problem, we are reduced to solve $\tilde{\mathbb{A}}\partial_{\tilde{x}}\tilde{q}-\tilde{\mathbb{B}}\tilde{q}=0$. In an harmonic form the problem becomes $(-ik\tilde{\mathbb{A}}-\tilde{\mathbb{B}})Q=0$. These solutions are the eleven eigenvectors of \ref{SystDiffAdim} corresponding  to the eigenvalue $\omega=0$. Instead of considering this algebraic problem, it is possible to consider the solution of the ordinary differential equation $\tilde{\mathbb{A}}\partial_{\tilde{x}}\tilde{q}-\tilde{\mathbb{B}}\tilde{q}=0$ due to some simplifications. Indeed it has been shown earlier $b_{i1}=0$. A second argument is that no-inertia intervenes and then $d_{ij}=0$ as this latter is related to micro-inertia. Accordingly the problem reduces to simple equalities:
%>>>>>>> 36bc2233a0358b090e9d33cf214d05ae557f0948
$$
\begin{array}{c}
\tilde{\rho}=\textit{cste}, \quad\quad	\tilde{v}_{1}=\textit{cste},
\\
a_{12}=-a_{21}, \quad\quad
a_{31}=-a_{13}
\end{array}
$$
and two uncoupled differential systems, listed hereafter:
\begin{equation}\label{DistortionIncompatibility1}
\begin{array}{rcl}
\pd_{x}b_{22}&=&-(\frac{\chi_{h}}{\chi_{\infty}})^2\left(a_{23}+a_{32}\right)\\
\pd_{x}b_{33}&=&+(\frac{\chi_{h}}{\chi_{\infty}})^2\left(a_{23}+a_{32}\right)
\end{array}
\end{equation}
%<<<<<<< HEAD
%Hence if $a_{23}=-a_{32}$, then $b_{22}$ and $b_{33}$ are constant, but if it's not the case, they 
%still vary in opposite manner, proportionally to this lake of skew-symmetry of the distorsion 
%tensor. \\
%According to (\ref{ABrelationAdim}), $b_{22}=-\pd_{\tilde{x}}\tilde{a}_{23}$ and 
%$b_{33}=\pd_{\tilde{x}}\tilde{a}_{32}$. Injecting these relations in 
%(\ref{DistortionIncompatibility1}):
%=======
\begin{equation}\label{StationnnaryMotion1}
\begin{array}{rcl}
\tilde{a}_{11}&=&\frac{1}{2}\left(\tilde{a}_{22}+\tilde{a}_{33}\right)  \\[6pt]
 \chi_p \tilde{b}_{23}+\partial_{x}\tilde{v}_2&=&0 \\
\chi _{\infty }^2\partial_{x}\tilde{b}_{23} &=&\chi _h^2 \left(\tilde{a}_{22}-\tilde{a}_{33}\right)\\[6pt]
-\chi_p \tilde{b}_{32}+\partial_{x}\tilde{v}_3&=&0 \\
\chi _{\infty }^2\partial_{x}\tilde{b}_{32}&=&\chi _h^2 \left(\tilde{a}_{22}-\tilde{a}_{33}\right)
\end{array}
\end{equation}   
The system (\ref{DistortionIncompatibility1}) couples distortion and incompatibility whereas (\ref{StationnnaryMotion1}) is related to stationary motion at a macro-scale. 
%-------------------------------------
\subsection{Distortion and incompatibility coupling}
%-------------------------------------
Let consider the system (\ref{DistortionIncompatibility1}). According to (\ref{ABrelationAdim}), $b_{22}=-\partial_{\tilde{x}}\tilde{a}_{23}$ and $b_{33}=\partial_{\tilde{x}}\tilde{a}_{32}$. Injecting these relations in (\ref{DistortionIncompatibility1}):
%>>>>>>> 36bc2233a0358b090e9d33cf214d05ae557f0948
\begin{equation}\label{DistortionIncompatibility2}
\begin{array}{rcl}
-\pd_{\tilde{x}}^2\tilde{a}_{23}&=&-(\frac{\chi_{h}}{\chi_{\infty}})^2\left(a_{23}+a_{32}\right)\\
\pd_{\tilde{x}}^2\tilde{a}_{32}&=&+(\frac{\chi_{h}}{\chi_{\infty}})^2\left(a_{23}+a_{32}\right)
\end{array}
\end{equation}
%<<<<<<< HEAD
%Then 
%$$
%\pd_{\tilde{x}}^2(\tilde{a}_{23}+\tilde{a}_{32})-(\frac{\chi_{h}}{\chi_{\infty}})^2\left(a_{23}+a_{32}\right)=0
%$$
%then these non-diagonal termes are opposite up to an harmonic function 
%=======
Then combining these equations:
\begin{equation}
\begin{array}{rcl}
\partial_{\tilde{x}}^2(\tilde{a}_{23}-\tilde{a}_{32})&=&0\\
\partial_{\tilde{x}}^2(\tilde{a}_{23}+\tilde{a}_{32})&=&(\frac{\chi_{h}}{\chi_{\infty}})^2\left(a_{23}+a_{32}\right)
\end{array}
\end{equation}
then: 
\begin{equation}
\begin{array}{rcl}
\tilde{a}_{23}-\tilde{a}_{32}&=&2a_{0}+2a_{1}\tilde{x}
\\
\tilde{a}_{23}+\tilde{a}_{32}&=&2a_{s1}\sinh{(\frac{\chi_{h}}{\chi_{\infty}}\tilde{x}})+2a_{c1}\cosh{(\frac{\chi_{h}}{\chi_{\infty}}\tilde{x}}),
\end{array}
\end{equation}
We have in terms of distortion
\begin{equation}
\begin{array}{rcl}
\tilde{a}_{23}&=&+(a_{0}+a_{1}\tilde{x})+a_{s1}\sinh{(\frac{\chi_{h}}{\chi_{\infty}}\tilde{x}})+a_{c1}\cosh{(\frac{\chi_{h}}{\chi_{\infty}}\tilde{x}})
\\
\tilde{a}_{32}&=&-(a_{0}+a_{1}\tilde{x})+a_{s1}\sinh{(\frac{\chi_{h}}{\chi_{\infty}}\tilde{x}})+a_{c1}\cosh{(\frac{\chi_{h}}{\chi_{\infty}}\tilde{x}})
\end{array}
\end{equation}
and for the incompatibility:
%>>>>>>> 36bc2233a0358b090e9d33cf214d05ae557f0948
$$
\begin{array}{rcl}
\tilde{b}_{22}&=&a_{1}+(\frac{\chi_{h}}{\chi_{\infty}})\left(a_{s1}\cosh{(\frac{\chi_{h}}{\chi_{\infty}}\tilde{x}})+a_{c1}\sinh{(\frac{\chi_{h}}{\chi_{\infty}}\tilde{x}})\right)\\
\tilde{b}_{33}&=&-a_{1}+(\frac{\chi_{h}}{\chi_{\infty}})\left(a_{s1}\cosh{(\frac{\chi_{h}}{\chi_{\infty}}\tilde{x}})+a_{c1}\sinh{(\frac{\chi_{h}}{\chi_{\infty}}\tilde{x}})\right)
\end{array}
$$
Here $a_{0}$, $a_{1}$, $a_{c}$ and $a_{s}$ are defined by boundary conditions. The coefficient 2 has been introduced for convenience.  If $a_{c}=0$ and $a_{s}=0$, we have $\tilde{a}_{23}=-\tilde{a}_{32}$ (like for the other non-diagonal terms of $\tilde{\boldsymbol{a}}$) and $b_{22}=-b_{33}=a_{1}$. 

%-------------------------------------
\subsection{Stationary motion}
%-------------------------------------
According to (\ref{ABrelationAdim}) we have $\tilde{b}_{23}= \partial_{\tilde{x}}\tilde{a}_{22}$ and $\tilde{b}_{32}=- \partial_{\tilde{x}}\tilde{a}_{33}$. The system is then 
\begin{equation}\label{StationnnaryMotion2}
\begin{array}{rcl}
%<<<<<<< HEAD
%\tilde{a}_{11}&=&\frac{1}{2}\left(\tilde{a}_{22}+\tilde{a}_{33}\right)  \\[8pt]
% \chi_p \tilde{b}_{23}+\tilde{d}_{21}+\pd_{x}\tilde{v}_2&=&0 \\
%\chi _{\infty }^2\pd_{x}\tilde{b}_{23}+\chi _p \pd_{x}\tilde{d}_{21} &=&\chi _h^2 
%\left(\tilde{a}_{22}-\tilde{a}_{33}\right)\\[8pt]
%-\chi_p \tilde{b}_{32}+\tilde{d}_{31}+\pd_{x}\tilde{v}_3&=&0 \\
%- \chi _{\infty }^2\pd_{x}\tilde{b}_{32}+\chi _p \pd_{x}\tilde{d}_{31}&=&\chi _h^2 
%\left(\tilde{a}_{33}-\tilde{a}_{22}\right)
%=======
\tilde{a}_{11}&=&\frac{1}{2}\left(\tilde{a}_{22}+\tilde{a}_{33}\right)  \\[6pt]
 \chi_p \partial_{\tilde{x}}\tilde{a}_{22}+\partial_{x}\tilde{v}_2&=&0 \\
\chi _{\infty }^2 \partial_{\tilde{x}}^2\tilde{a}_{22} &=&\chi _h^2 \left(\tilde{a}_{22}-\tilde{a}_{33}\right)\\[6pt]
 \chi_p \partial_{\tilde{x}}\tilde{a}_{33}+\partial_{x}\tilde{v}_3&=&0 \\
\chi _{\infty }^2 \partial_{\tilde{x}}^2\tilde{a}_{33} &=&\chi _h^2 \left(\tilde{a}_{33}-\tilde{a}_{22}\right)
%>>>>>>> 36bc2233a0358b090e9d33cf214d05ae557f0948
\end{array}
\end{equation}  
Combining the lines 3 and 5 we have
$$
\begin{array}{rcl}
%<<<<<<< HEAD
% \chi_p \tilde{b}_{23}+\tilde{d}_{21}+\pd_{x}\tilde{v}_2&=&0 \\
%\chi _{\infty }^2\pd_{x}\tilde{b}_{23}+\chi _p \pd_{x}\tilde{d}_{21} &=&0\\[8pt]
%=======
\partial_{\tilde{x}}^2(\tilde{a}_{22}+\tilde{a}_{33})&=&0\\
 \partial_{\tilde{x}}^2(\tilde{a}_{22}-\tilde{a}_{33}) &=&2(\frac{\chi _h}{\chi _{\infty }})^2 \left(\tilde{a}_{22}-\tilde{a}_{33}\right) 
%>>>>>>> 36bc2233a0358b090e9d33cf214d05ae557f0948
\end{array}
 $$
in other words
$$
\begin{array}{rcl}
\tilde{a}_{22}+\tilde{a}_{33}&=&2(a_{2}+a_{3} \tilde{x})\\
\tilde{a}_{22}-\tilde{a}_{33}&=&2\left(a_{c2} \cosh \left(\frac{\sqrt{2}  \chi _h}{\chi
   _{\infty }}\tilde{x}\right)+a_{s2} \sinh \left(\frac{\sqrt{2} \chi _h}{\chi _{\infty
   }}\tilde{x} \right)\right)
\end{array}
$$
where again $a_{2}$, $a_{3}$, $a_{c}$ and $a_{s}$ are some constants defined by boundary condition if necessary\footnote{They are independ of the constant presented in the previous sub-section but we have used the same dimension in order to avoid additional notations.}. Finally:
$$
\begin{array}{rcl}
a_{22}(\tilde{x}) &=&a_{2}+a_{3} \tilde{x}+a_{c2} \cosh (\frac{\sqrt{2}  \chi _h}{\chi
   _{\infty }}\tilde{x})+a_{s2} \sinh (\frac{\sqrt{2} \chi _h}{\chi _{\infty
   }}\tilde{x} ),
\\
a_{33}(\tilde{x}) &=&a_{2}+a_{3} \tilde{x}-a_{c2} \cosh (\frac{\sqrt{2}\chi _h}{\chi
   _{\infty }} \tilde{x} )-a_{s2} \sinh (\frac{\sqrt{2} \chi _h}{\chi _{\infty
   }}\tilde{x} )
\end{array} 
$$
%<<<<<<< HEAD
%According to (\ref{ABrelationAdim}), $\tilde{b}_{23}= \pd_{\tilde{x}}\tilde{a}_{22}$:
%$$
%\begin{array}{rcl}
% \chi_p\pd_{\tilde{x}}\tilde{a}_{22}+\tilde{d}_{21}+\pd_{x}\tilde{v}_2&=&0 \\
%\chi _{\infty }^2\pd_{\tilde{x}}^2\tilde{a}_{22}+\chi _p \pd_{x}\tilde{d}_{21} &=&0\\[8pt]
%=======
Now we observe that transverse velocity fields are completely defined:
$$
\begin{array}{rcl}
\tilde{v}_{2}(\tilde{x}) &=&v_{20}-\chi _p \left(a_{3} \tilde{x}+a_{c2} \cosh (\frac{\sqrt{2} 
   \chi _h}{\chi _{\infty }}\tilde{x})+a_{s2} \sinh (\frac{\sqrt{2} \chi _h}{\chi
   _{\infty }}\tilde{x}) \right)
 \\
\tilde{v}_{3}(\tilde{x}) &=&v_{30}-\chi _p \left(a_{3} \tilde{x}-a_{c2} \cosh
   (\frac{\sqrt{2}  \chi _h}{\chi _{\infty }}\tilde{x})-a_{s2} \sinh
   (\frac{\sqrt{2}  \chi _h}{\chi _{\infty }}\tilde{x})\right)
\end{array} 
$$
where $v_{20}$ and $v_{30}$ are integration constants.

%-------------------------------------
\subsection{Analysis of the distorsion}
%-------------------------------------
Let introduce
$$
\begin{array}{rcl}
f_{1}(\tilde{x})	&=&	a_{c1}\cosh{(\frac{\chi_{h}}{\chi_{\infty}}\tilde{x}})+a_{s1}\sinh{(\frac{\chi_{h}}{\chi_{\infty}}\tilde{x}})\\
f_{2}(\tilde{x})	&=&	a_{c2} \cosh (\frac{\sqrt{2} 
   \chi _h}{\chi _{\infty }}\tilde{x})+a_{s2} \sinh (\frac{\sqrt{2} \chi _h}{\chi
   _{\infty }}\tilde{x}) 
%>>>>>>> 36bc2233a0358b090e9d33cf214d05ae557f0948
\end{array}
$$
the symmetric and skew-symmetric part of the distortion are
$$
%<<<<<<< HEAD
%\tilde{d}_{21}=-\frac{\chi _{\infty }^2}{\chi _p}\pd_{\tilde{x}}\tilde{a}_{22},
%\quad\quad\quad
%\tilde{v}_2 =\frac{\chi _{\infty}^2-\chi _p^2}{\chi _p}\tilde{a}_{22}
%$$
%For the problem in the $(\boldsymbol{e_{1}},\boldsymbol{e_{3}})$-plane, we have according to 
%(\ref{ABrelationAdim}) $\tilde{b}_{32}=- \pd_{\tilde{x}}\tilde{a}_{33}$. After computation we 
%obtain:
%$$
%\tilde{d}_{31}=-\frac{\chi _{\infty }^2}{\chi _p}\pd_{\tilde{x}}\tilde{a}_{33},
%\quad\quad\quad
%\tilde{v}_3 =\frac{\chi _{\infty}^2-\chi _p^2}{\chi _p}\tilde{a}_{33}
%=======
a^{S}=\begin{pmatrix}
\frac{a_{0}+a_{2}+(a_{1}+a_{3})\tilde{x}}{2}&0&0\\
0	&a_{2}+a_{3}\tilde{x}+f_{2}(\tilde{x})&f_{1}(\tilde{x})\\
0	&f_{1}(\tilde{x})		&	a_{0}+a_{1}\tilde{x}-f_{2}(\tilde{x})&\end{pmatrix}
$$
$$
a^{A}=\begin{pmatrix}
0&a_{12}&a_{13}\\
-a_{12}&0	&a_{0}+a_{1}\tilde{x}\\
-a_{13}&-a_{0}-a_{1}\tilde{x}	&0\end{pmatrix}
%>>>>>>> 36bc2233a0358b090e9d33cf214d05ae557f0948
$$
%-------------------------------------
\subsection{Displacement}
%-------------------------------------
We have
$$
%<<<<<<< HEAD
%\pd_{\tilde{x}}\tilde{d}_{21}=-\pd_{\tilde{x}}\tilde{d}_{31}=\frac{1}{\chi_{p}}\left( 
%2\chi_{h}^2a_{22}-\chi_{\infty}^2\pd_{\tilde{x}}^2a_{22}\right)
%$$ 
%\end{itemize}
%
%=======
a_{ij}=\frac{\partial u_{i}}{\partial x_{j}}
\quad\quad
v_{i}=\frac{\partial u_{i}}{\partial t}
$$
Let simplify notation as 
$$
\begin{array}{rcl}
\tilde{v}_{2}(\tilde{x}) &=&v_{20}-\chi _p f_{2}(\tilde{x})
 \\
\tilde{a}_{22}(\tilde{x}) &=&a_{0}+ f_{2}(\tilde{x})
\\
\tilde{v}_{3}(\tilde{x}) &=&v_{30}-\chi _p f_{3}(\tilde{x})
 \\
\tilde{a}_{33}(\tilde{x}) &=&a_{0}+ f_{3}(\tilde{x})
\end{array} 
\quad \textrm{with}
\quad\quad
\begin{array}{rcl}
f_{2}(\tilde{x})&=&a_{1} \tilde{x}-a_{c} \cosh
   (\frac{\sqrt{2}  \chi _h}{\chi _{\infty }}\tilde{x})-a_{s} \sinh
   (\frac{\sqrt{2}  \chi _h}{\chi _{\infty }}\tilde{x})
   \\
f_{3}(\tilde{x})&=&a_{1} \tilde{x}+a_{c} \cosh
   (\frac{\sqrt{2}  \chi _h}{\chi _{\infty }}\tilde{x})+a_{s} \sinh
   (\frac{\sqrt{2}  \chi _h}{\chi _{\infty }}\tilde{x})
   \end{array} 
$$
We have after time integration of $\tilde{v}_{2}(\tilde{x}) $ and $\tilde{v}_{3}(\tilde{x})$ :
$$
\begin{array}{rcl}
u_{2}(\tilde{x},\tilde{t})=\left(v_{20}-\chi _p f_{2}(\tilde{x})\right)\tilde{t} +g_{2}(\tilde{x},x_{2},x_{3})
\\
u_{3}(\tilde{x},\tilde{t})=\left(v_{30}-\chi _p f_{3}(\tilde{x})\right)\tilde{t} +g_{3}(\tilde{x},x_{2},x_{3})
   \end{array} 
$$
But due to definition of $\tilde{a}_{22}(\tilde{x})$ and $\tilde{a}_{33}(\tilde{x})$:
$$
\frac{\partial g_{2}}{\partial x_{2}}(\tilde{x},x_{2},x_{3}) = a_{0}+f_{2}(\tilde{x})
$$
then 
$$
 g_{2}(\tilde{x},x_{2},x_{3}) = \left(a_{0}+f_{2}(\tilde{x})\right) x_{2} + h_{2}(x_{3})
$$
\textcolor{red}{TO BE CONTINUED}
%>>>>>>> 36bc2233a0358b090e9d33cf214d05ae557f0948
%-------------------------------------
\subsection{Elastic problem}
%-------------------------------------

In the special situation where both $\chi_{p}$ and $\chi_{h}$ are zset to ero, we have
$$
\begin{array}{c}
\mathcal{D}_{3}=\omega^2\, \left(\omega^2-2\chi_{h}^2\right)\, \left(\omega^2-\chi_{h}^2(2+k^2)\right), \\[6pt]
\mathcal{D}_{2}=\omega^4 \, \left(\omega^2-\chi_{h}^2(2+k^2)\right) \, \left( 
 3 \omega^2(\omega^2 - k^2)+ \chi_{h}^2( k^2 (6 - 4 \omega^2)-6 \omega^2)  \right)
 \end{array}
$$

\section{Small amplitude waves}

For the plane-wave analysis it is sufficient to consider the nonlinear model~\eqref{PDE} in the 
small deformation limit, i.e. we assume that the distortion field can be decomposed as $ \Dist = 
\bm{I} + \Distsmall $, with $ \bm{I} $ being the identity matrix and $ \Distsmall $ being the 
non-symmetric small strain tensor $ \Distsmall \neq \Distsmall^\transpose $.



\printbibliography

\end{document}
% Acta Mechanica
% Archive of Applied Mechanics
% Mathematics and Mechanics of Solids
% CMAT