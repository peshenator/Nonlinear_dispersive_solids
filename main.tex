%%%%%%%%%%%%%%%%%%%%%%%%%%%%%%%%%%%%%%%%%
% Arsclassica Article
% LaTeX Template
% Version 1.1 (1/8/17)
%
% This template has been downloaded from:
% http://www.LaTeXTemplates.com
%
% Original author:
% Lorenzo Pantieri (http://www.lorenzopantieri.net) with extensive modifications by:
% Vel (vel@latextemplates.com)
%
% License:
% CC BY-NC-SA 3.0 (http://creativecommons.org/licenses/by-nc-sa/3.0/)
%
%%%%%%%%%%%%%%%%%%%%%%%%%%%%%%%%%%%%%%%%%

%----------------------------------------------------------------------------------------
%	PACKAGES AND OTHER DOCUMENT CONFIGURATIONS
%----------------------------------------------------------------------------------------

\documentclass[
10pt, % Main document font size
a4paper, % Paper type, use 'letterpaper' for US Letter paper
oneside, % One page layout (no page indentation)
%twoside, % Two page layout (page indentation for binding and different headers)
headinclude,footinclude, % Extra spacing for the header and footer
%BCOR5mm, % Binding correction
table
]{scrartcl}

%%%%%%%%%%%%%%%%%%%%%%%%%%%%%%%%%%%%%%%%%
% Arsclassica Article
% Structure Specification File
%
% This file has been downloaded from:
% http://www.LaTeXTemplates.com
%
% Original author:
% Lorenzo Pantieri (http://www.lorenzopantieri.net) with extensive modifications by:
% Vel (vel@latextemplates.com)
%
% License:
% CC BY-NC-SA 3.0 (http://creativecommons.org/licenses/by-nc-sa/3.0/)
%
%%%%%%%%%%%%%%%%%%%%%%%%%%%%%%%%%%%%%%%%%

%----------------------------------------------------------------------------------------
%	REQUIRED PACKAGES
%----------------------------------------------------------------------------------------

\usepackage[
nochapters, % Turn off chapters since this is an article        
beramono, % Use the Bera Mono font for monospaced text (\texttt)
eulermath,% Use the Euler font for mathematics
pdfspacing, % Makes use of pdftex’ letter spacing capabilities via the microtype package
dottedtoc % Dotted lines leading to the page numbers in the table of contents
]{classicthesis} % The layout is based on the Classic Thesis style



\usepackage[hmarginratio=1:1,top=25mm,left=20mm,columnsep=25pt]{geometry}
\usepackage{relsize} % e.g. used for \mathsmaller
\usepackage{bm}

\usepackage{arsclassica} % Modifies the Classic Thesis package

\usepackage[T1]{fontenc} % Use 8-bit encoding that has 256 glyphs

\usepackage[utf8]{inputenc} % Required for including letters with accents

\usepackage{graphicx} % Required for including images
\graphicspath{{Figures/}} % Set the default folder for images

\usepackage{enumitem} % Required for manipulating the whitespace between and within lists

\usepackage{lipsum} % Used for inserting dummy 'Lorem ipsum' text into the template

\usepackage{subfig} % Required for creating figures with multiple parts (subfigures)

\usepackage{amsmath,amssymb,amsthm} % For including math equations, theorems, symbols, etc

\usepackage{varioref} % More descriptive referencing

\usepackage{accents}

\usepackage[bottom]{footmisc}

\usepackage{titling} % it needs to define \thanksmarkseries

%----------------------------------------------------------------------------------------
% NEW COMMANDS
%----------------------------------------------------------------------------------------
\renewcommand{\AA}{\bm{A}}
\newcommand{\GG}{\bm{G}}
\newcommand{\MM}{\bm{M}}
\newcommand{\Id}{\bm{I}}
\newcommand{\vv}{\bm{v}}
\newcommand{\QQ}{\bm{Q}}
\renewcommand{\SS}{\bm{S}}
\newcommand{\Lie}{\mathfrak{L}}
\newcommand{\calE}{\mathcal{E}}						%

\newcommand{\IP}[1]{{\color{Red}IP:\ \ #1}}
\newcommand{\ER}[1]{{\color{Green}ER:\ \ #1}}
\newcommand{\BL}[1]{{\color{Cerulean}BL:\ \ #1}}
\newcommand{\NF}[1]{{\color{Plum}NF:\ \ #1}}


\newcommand{\sA}{\mathsmaller A}
\newcommand{\sB}{\mathsmaller B}
\newcommand{\sC}{\mathsmaller C}
\newcommand{\sD}{\mathsmaller D}
\newcommand{\sM}{\mathsmaller M}
\newcommand{\sN}{\mathsmaller N}
\newcommand{\sL}{\mathsmaller L}
\newcommand{\kronecker}[2]{\delta^{#1}_{\phantom{#1}#2}}
\newcommand{\durg}[2]{ D_{#1}^{\phantom{#1}#2} }	% Eulerian components of 
%the D field
\newcommand{\LeviCivitaUp}[1]{\varepsilon^{#1}}

\newcommand{\pd}{\partial}
\newcommand{\F}[2]{F^{\ #1}_{\mathsmaller#2}}
\newcommand{\hatF}[2]{\hat{F}^{\ #1}_{\mathsmaller#2}}
\newcommand{\A}[2]{A^{\mathsmaller#1}_{\ #2}}

\newcommand{\dist}[2]{ A^{#1}_{\phantom{#1}#2} }	% Component of the 
\newcommand{\Stress}[2]{ \Sigma^{#1}_{\phantom{#1}#2} }	% Total stress tensor
%distortion field A
\newcommand{\distsmall}[2]{ a_{{#1}{#2}} }	% Small distortion field A 
\newcommand{\Dist}{ \bm{A} }	% Distortion field A in matrix notations
\newcommand{\Burg}{ \bm{B} }	% Burgers tensor = curl(A)
\newcommand{\Durg}{ \bm{D} }	% Complimentary to the Burgers
\newcommand{\Distsmall}{ \bm{a} }	% Small distortion field A in matrix 
%notations
\newcommand{\Plastsmall}{ \bm{\pi} }	% Small distortion field A in matrix 
%notations
\newcommand{\Defgrad}{ \bm{F} }
\newcommand{\iDist}{ \bm{E} }
\newcommand{\symA}{\text{sym}(\bm{a})}
\newcommand{\skewA}{\text{skew}(\bm{a})}
\newcommand{\symP}{\text{sym}(\bm{\Plastsmall})}
\newcommand{\burg}[2]{ B^{{#1}{#2}} }	% Eulerian components of the B field
\newcommand{\itetr}[2]{e^{\phantom{#2}#1}_{#2}}
\newcommand{\tetr}[2]{a^{#1}_{\phantom{#1}#2}}
\newcommand{\rtetr}[2]{a^{#1}_{(\text{r}) #2}}
\newcommand{\spin}[2]{\omega^{#1}_{\phantom{#1}#2}}
\newcommand{\Lor}[2]{\Lambda^{#1'}_{\phantom{#1}#2}}
\newcommand{\iLor}[2]{\Lambda^{\phantom{#2}#1}_{#2'}}
\newcommand{\vel}[1]{v^{#1}}
\newcommand{\D}[1]{\mathcal{D}_{#1}} % Fock-Ivanencko cov derivative
\newcommand{\Tors}[2]{T^{#1}_{\phantom{a}#2}}
\newcommand{\Supp}[2]{S_{#1}^{\phantom{a}#2}}	%supepotential
\newcommand{\Torsl}[1]{T_{#1}}
\newcommand{\ET}[2]{E^{#1}_{\phantom{#1}#2}}	%Torsion decomposition, analog 
%of Electric field
\newcommand{\eT}[2]{D_{#1}^{\phantom{#1}#2}}	%Torsion decomposition, analog 
%of Electric field
\newcommand{\BT}[2]{B^{#1#2}}	%Torsion decomposition, analog of Magnetic field
\newcommand{\hT}[2]{H^{#1#2}}	%Torsion decomposition, analog of Magnetic field
\newcommand{\W}[2]{\mathcal{W}^{#1}_{\phantom{#1}#2}}
\newcommand{\w}[2]{W^{#1}_{\phantom{#1}#2}}
\newcommand{\FI}{Fock-Ivanenko}
\newcommand{\We}{Weitzenb\"ock}
\newcommand{\Lag}{\mathcal{L}}	% Lagrangian which depends on ordinary 
%derivatives
\newcommand{\Lagcov}{\pounds}% Lagrangian which depends on gauge covariant 
%derivatives
\newcommand{\Laghodge}{L}% Lagrangian which depends on the Hodge dual of the 
%torsion
\newcommand{\Lagtors}{\mathbb{L}}% Lagrangian which depends on torsion
\newcommand{\LagBE}{\mathfrak{L}}% Lagrangian which depends on the B and E 
%fields
\newcommand{\veps}{\varepsilon}
\newcommand{\EM}[2]{\Sigma^{#1}_{\phantom{#1}#2}}
\newcommand{\transpose}{{\mathrm {\mathsmaller T}}}
\newcommand{\tr}{\text{tr}}

\newcommand{\tegr}{TEGR}
\newcommand{\HT}[1]{\accentset{\star}{T}^{#1}}

\newcommand{\csh}{c_\text{sh}}	% 	shear sound speed
\newcommand{\csp}{c_\text{sp}}	% 	a sound speed related to the spin
\newcommand{\cinf}{c_\infty}	% 	a sound speed related to 
%1/\sqrt{\mu\eps}
%----------------------------------------------------------------------------------------
%	THEOREM STYLES
%---------------------------------------------------------------------------------------

\theoremstyle{definition} % Define theorem styles here based on the definition style (used for definitions and examples)
\newtheorem{definition}{Definition}

\theoremstyle{plain} % Define theorem styles here based on the plain style (used for theorems, lemmas, propositions)
\newtheorem{theorem}{Theorem}

\theoremstyle{remark} % Define theorem styles here based on the remark style (used for remarks and notes)

%----------------------------------------------------------------------------------------
%	HYPERLINKS
%---------------------------------------------------------------------------------------




%----------------------------------------------------------------------------------------
%	BIBLATEX
%---------------------------------------------------------------------------------------

\usepackage[backend=bibtex,giveninits=true,url=false,doi=true,eprint=true,isbn=false,
backref,backrefstyle=none,maxbibnames=99]{biblatex}
\DefineBibliographyStrings{english}{%
  backrefpage = {Cited on p\adddot},%
  backrefpages = {Cited on pp\adddot}%
}

\bibliography{library}

\renewcommand*{\bibfont}{\footnotesize}

% in order to suppress 'In:'
\renewbibmacro{in:}{%
  \ifboolexpr{%
     test {\ifentrytype{article}}%
  }{}{\printtext{\bibstring{in}\intitlepunct}}%
}

%----------------------------------------------------------------------------------------
% these commands allow to put equations in a fancy boxes:
%----------------------------------------------------------------------------------------
\usepackage{empheq}
\newlength\mytemplen
\newsavebox\mytempbox
\makeatletter
\definecolor{cream}{rgb}{.81, .88, 1}
 \newcommand\mycreambox{%
     \@ifnextchar[%]
        {\@mycreambox}%
        {\@mycreambox[0pt]}}
 \def\@mycreambox[#1]{%
     \@ifnextchar[%]
        {\@@mycreambox[#1]}%
        {\@@mycreambox[#1][0pt]}}
 \def\@@mycreambox[#1][#2]#3{
     \sbox\mytempbox{#3}%
     \mytemplen\ht\mytempbox
     \advance\mytemplen #1\relax
     \ht\mytempbox\mytemplen
     \mytemplen\dp\mytempbox
     \advance\mytemplen #2\relax
     \dp\mytempbox\mytemplen
     \colorbox{cream}{\hspace{1em}\usebox{\mytempbox}\hspace{1em}}}
 \makeatother

% ------------------------------------------------------------------------------
\newcommand*\samethanks[1][\value{footnote}]{\footnotemark[#1]}

 % Include the structure.tex file which specified the document structure and 
%layout

\PassOptionsToPackage{hyperfootnotes=true}{hyperref}
%[
%%draft, % Uncomment to remove all links (useful for printing in black and 
%%%white)
%colorlinks=true, 
%breaklinks=true, 
%bookmarks=true,
%bookmarksnumbered,
%urlcolor=webbrown, 
%linkcolor=RoyalBlue, 
%citecolor=webgreen, % Link colors
%pdftitle={}, % PDF title
%pdfauthor={\textcopyright}, % PDF Author
%pdfsubject={}, % PDF Subject
%pdfkeywords={}, % PDF Keywords
%pdfcreator={pdfLaTeX}, % PDF Creator
%pdfproducer={LaTeX with hyperref and ClassicThesis}, % PDF producer
%hyperfootnotes=true
%]

%----------------------------------------------------------------------------------------
%	TITLE AND AUTHOR(S)
%----------------------------------------------------------------------------------------

\title{\large\normalfont\spacedallcaps{Modeling phononic band gap in the 
Riemann-Cartan geometry framework}} % The article 
%title

%\subtitle{Subtitle} % Uncomment to display a subtitle

\author{
%	author1:
\normalsize\textsc{Ilya Peshkov}\thanks{Paul Sabatier 
University, IMT, Toulouse, France},\ \ 
%	author2:
\normalsize\textsc{Lo\"ic Le Marrec}\thanks{Université de 
	Rennes I, IRMAR, Rennes, France},\ \ 
%	author3:
\normalsize\textsc{Van Hoi Nguyen(?)}\samethanks[2],\ \
%	author4:
\normalsize\textsc{Evgeniy Romenski(?)}\thanks{Sobolev 
	Institute of Mathematics, Novosibirsk, Russia}$\ \, ^, $\thanks{Novosibirsk 
	State University, Novosibirsk, Russia}
%\ldots 
}
\thanksmarkseries{arabic}
% The article author(s) - author afiliations 
%need to be 
%specified in the 
%AUTHOR AFFILIATIONS block

\date{\small\today} % An optional date to appear under the author(s)

%----------------------------------------------------------------------------------------


\begin{document}

%----------------------------------------------------------------------------------------
%	HEADERS
%----------------------------------------------------------------------------------------

\renewcommand{\sectionmark}[1]{\markright{\spacedlowsmallcaps{#1}}} % The header for all pages 
%(oneside) or for even pages (twoside)
%\renewcommand{\subsectionmark}[1]{\markright{\thesubsection~#1}} % Uncomment when using the 
%%twoside option - this modifies the header on odd pages
\lehead{\mbox{\llap{\small\thepage\kern1em\color{halfgray} 
\vline}\color{halfgray}\hspace{0.5em}\rightmark\hfil}} % The header style

\pagestyle{scrheadings} % Enable the headers specified in this block

%----------------------------------------------------------------------------------------
%	TABLE OF CONTENTS & LISTS OF FIGURES AND TABLES
%----------------------------------------------------------------------------------------

\maketitle % Print the title/author/date block

\setcounter{tocdepth}{2} % Set the depth of the table of contents to show sections and subsections 
%only

\tableofcontents % Print the table of contents

% \listoffigures % Print the list of figures

% \listoftables % Print the list of tables

%----------------------------------------------------------------------------------------
%	ABSTRACT
%----------------------------------------------------------------------------------------

\section*{Abstract} % This section will not appear in the table of contents due to the star 
% (\section*)
Modeling of acoustic waves in dispersive solids is discussed in the context of 
the Riemann-Cartan geometry. We propose a continuum model which is formulated 
in terms of the main objects of the Riemann-Cartan geometry such as the 
non-holonomic triad (distortion field) and torsion field. It is 
demonstrated that the model is able to describe the phononic band gap in 
dispersive solids. 

%----------------------------------------------------------------------------------------
%	AUTHOR AFFILIATIONS
%----------------------------------------------------------------------------------------
%\let\thefootnote\relax\footnotetext{* \textit{peshenator@gmail.com}}
%\let\thefootnote\relax\footnotetext{\textsuperscript{1} \textit{Paul Sabatier 
%University, IMT, Toulouse, France}}
%\let\thefootnote\relax\footnotetext{\textsuperscript{2} \textit{Université de 
%Rennes I, IRMAR, Rennes, France}}
%\let\thefootnote\relax\footnotetext{\textsuperscript{3} \textit{Sobolev 
%Institute of Mathematics, Novosibirsk, Russia}}
%\let\thefootnote\relax\footnotetext{\textsuperscript{4} \textit{Novosibirsk 
%State University, Novosibirsk, Russia}}
%\let\thefootnote\relax\footnotetext{\textsuperscript{4} \textit{Aix-Marseille 
%Université, IUSTI,  Marseille, France}}
%\let\thefootnote\relax\footnotetext{\textsuperscript{4} \textit{CNRS, LMA, 
%Marseille, France}}
%----------------------------------------------------------------------------------------

%\newpage % Start the article content on the second page, remove this if you have a longer abstract 
%that goes onto the second page

% PARAGRAPH OPTIONS:
\setlength\parindent{10pt} % sets indent to zero
\setlength{\parskip}{5pt} % changes vertical space between paragraphs
% PARAGRAPH OPTIONS.

%----------------------------------------------------------------------------------------
%	INTRODUCTION
%----------------------------------------------------------------------------------------

\section{Introduction}

\section{Governing Equations}

In the absence of irreversible effects due to plasticity or viscosity, the 
governing equations derived in \cite{PRD-Torsion2019} read 
\begin{subequations}\label{PDE}
	\begin{align}
	&\frac{\pd \rho}{\pd t} + \frac{\pd(\rho \vel{k})}{\pd x^k} = 
	0, \label{PDE.extend.rho}
	\\[2mm]
%
	&\frac{\pd M_i}{\pd t} + \frac{ \pd }{\pd x^k}  \left( M_i \vel{k} + P 
	\kronecker{k}{i} + 
	\dist{a}{i}
	\calE_{\dist{a}{k}} - \burg{a}{k} \calE_{\burg{a}{ i}} - \durg{a}{k} 
	\calE_{\durg{a}{ i}}
	\right ) = 0,\label{PDE.extend.M}
	\\[2mm]
%
	&\frac{\pd \dist{a}{k}}{\pd t} +\frac{\pd (  {\dist{a}{i} \vel{i}}  )}{\pd 
	x^k} + 
	\vel{j} \left(\frac{\pd \dist{a}{k}}{\pd x^j} - \frac{\pd\dist{a}{j}}{\pd 
	x^k}\right) = 
	-\frac{1}{\alpha} \calE_{\durg{a}{k}},\label{PDE.extend.A}
	\\[2mm]
%	
	&\frac{\pd \burg{a}{i}}{\pd t} + \frac{\pd}{\pd x^k} \left(
	\burg{a}{i} \vel{k} - \vel{i} \burg{a}{k} + \LeviCivitaUp{ikj} 
	\calE_{\durg{a}{j}}
	\right) + \vel{i} \frac{\pd \burg{a}{k}}{\pd x^k} = 0,\label{PDE.extend.B}
	\\[2mm]
%	
	&\frac{\pd \durg{a}{i}}{\pd t} + \frac{\pd} {\pd x^k} \left( \durg{a}{i} 
	\vel{k}  -  \vel{i} 
	\durg{a}{k} - \LeviCivitaUp{i k j}  \calE_{\burg{a}{j}} \right) + \vel{i} 
	\frac{\pd 
		\durg{a}{k}}{\pd x^k}  = \frac{1}{\alpha}\calE_{\dist{a}{i}}, 
		\label{PDE.extend.D}
	\\[2mm]
%	
	&\frac{\pd s}{\pd t} + \frac{\pd (s \vel{k})}{\pd x^k} = 0. 
	\label{PDE.extend.s}
	\end{align}
\end{subequations}

The following energy conservation law can be obtained as the consequence of the 
PDEs \eqref{PDE}
\begin{equation}\label{energy.law}
\frac{\pd \calE}{\pd t} + \frac{\pd }{\pd x_k} \left( \vel{k} \calE + \vel{i} 
\Stress{k}{i} + \LeviCivitaUp{ijk} 
\calE_{\durg{a}{i}}\calE_{\burg{a}{j}}\right) = 0,
\end{equation}
where $ \dist{a}{k} $ is 
the field of bases triads, non-holonomic in general, (also called the 
distortion field in 
\cite{PRD-Torsion2019,DPRZ2016}). The fields $ \burg{a}{i} $ and $ \durg{a}{i} 
$ can be viewed as two parts of the four-torsion $ \Tors{a}{\mu\nu} 
= \pd_\mu \dist{a}{\nu} - \pd_\nu \dist{a}{\mu} $, $ \mu,\nu=0,1,2,3 $, see 
details in 
\cite{PRD-Torsion2019}, so that $ \burg{a}{i} = \alpha
\LeviCivitaUp{ijk}\pd_j\dist{a}{k}$ contains only spatial derivatives of $ 
\dist{a}{k} $ with $ \alpha \sim L^{-1}$ being the scaling constant and $ L $ 
is a length unit, while $ \durg{a}{i} $ contains both time and space 
derivatives\footnote{Its 
explicit 
expression in terms of $ 
\pd_t 
\dist{a}{k} $ and $ \pd_i \dist{a}{k} $ is rather impossible because it 
involves Legendre 
transformations which are non-linear.}. The physical role of $ \burg{a}{i} $ is 
to represent the small scale structural incompatibility in the distortion 
field, while $ \durg{a}{i} $ represents the micro inertial effect of the 
microstructure.

Also, $ \rho $ is the mass density which is constrained by $ \rho = \rho_0 
\det(\dist{a}{k}) $ with $ \rho_0 $ being the reference mas density, $ \vel{i} 
$ is the velocity 
of the medium, $ 
M_i $ is the total momentum 
of the medium which includes not only the 
matter momentum but also it includes a contribution due to 
torsion. Its specification depends on the energy potential $ \calE $ and is
constraint by the relation $ \calE_{M_i} = \vel{i} $. Also, 
\begin{equation}\label{stress}
\Stress{k}{i} := -
P \kronecker{k}{i} - \dist{a}{i} \calE_{\dist{a}{k}} + 
\burg{a}{k} 
\calE_{\burg{a}{i}} + \durg{a}{k}\calE_{\durg{a}{i}}  
\end{equation}is the total stress 
tensor, $ P := \rho \calE_\rho + s\calE_s + M_i \calE_{M_i} + 
\burg{a}{i}\calE_{\burg{a}{i}} + 
\durg{a}{i}
\calE_{\durg{a}{i}} - \calE $ is the thermodynamic pressure, and the last term 
in the energy flux, $ \LeviCivitaUp{ijk} 
\calE_{\durg{a}{i}}\calE_{\burg{a}{j}} $, is the contribution due 
to torsion.

Because equation \eqref{PDE.extend.M} represents the conservation law for the 
total (matter + torsion) momentum, in order to satisfy the conservation law 
of the angular momentum, the total momentum flux $ M_i \vel{k} - \Stress{k}{i} 
$ has to be symmetric, see details in \cite{PRD-Torsion2019}. This, however, 
cannot be guarantied for arbitrary energy potential $ \calE $ and imposes 
certain constraints on the choice of $ \calE $. Thus, an example for $ \calE $ 
which provides the symmetric momentum flux was proposed in 
\cite{PRD-Torsion2019}
\begin{subequations}\label{EOS}
\begin{equation}
\calE = \rho \veps(\rho,s) + \rho \frac{\csh^2}{4} ||\GG'||^2 + 
\frac{1}{2\rho} \MM^2 +
\calE^t(\MM,\Burg,\Durg)
\end{equation}
where $ \bm{G} =\AA^\transpose\AA$, and $ \bm{G}' = \bm{G} -
\frac{\text{tr}(\bm{G})}{3}\bm{I} $ is the deviatoric part of $ \GG $, while
\begin{equation}\label{energy.torsion}
%{\red \frac12 \rho\, c_t^2 ||\skewA||^2} + 
\calE^t  = \frac{1}{2}\left (\frac1\epsilon \, 
||\Durg||^2 
+ 
\frac1\mu \, ||\Burg||^2\right ) 
%
-\frac{1}{\rho}\sum_{a=1}^{3}\left|
\begin{array}{ccc}
M_1 & \durg{a}{1} & \burg{a}{1} \\
M_2 & \durg{a}{2} & \burg{a}{2} \\
M_3 & \durg{a}{3} & \burg{a}{3}
\end{array}
\right|
%
- \csp\sum_{a=1}^{3}\left|
\begin{array}{ccc}
\dist{a}{1} & \durg{a}{2} & \burg{a}{1} \\
\dist{a}{2} & \durg{a}{2} & \burg{a}{2} \\
\dist{a}{3} & \durg{a}{3} & \burg{a}{3}
\end{array}
\right|
\end{equation}
\end{subequations}
Here, 
$ \epsilon $ and $ \mu $ are torsion related transport parameters 
which together scale as the inverse velocity square, $ (\epsilon\mu)^{-1}\sim 
v^2 $ so that we introduce the velocity
\begin{equation}\label{light.speed}
\cinf = \frac{1}{\sqrt{\epsilon\mu}}.
\end{equation}
As we shall see later, $ \cinf $ is the maximum velocity at which mechanical 
perturbations can propagate in the medium.

The parameter $ \csp $ has the dimension of velocity and characterizes 
the propagation of rotational degrees of freedom. Also, $ \csh $ is the sound 
speed of propagation of tangential perturbations. Both $ \csp $ and $ \csh $ 
are constant in this paper. Finally, there is a bulk 
sound speed $ c_0 $ associated with the hydrodynamic part of the equation of 
state, that is $ \varepsilon(\rho,s) $, and so that the longitudinal sound 
speed $ c_\text{l} $ is expressed conventionally as
\begin{equation}\label{long.sound}
c_\text{l}^2 = c_0^2 + \frac43 \csh^2. 
\end{equation}
Therefore, we distinguish the following  characteristic velocities
\begin{equation}\label{key}
c_0, \qquad \csh, \qquad c_\text{l} = \sqrt{c_0^2 + \frac43 \csh^2}, 
\qquad \csp, \qquad \cinf = \frac{1}{\sqrt{\epsilon\mu}}.
\end{equation}

Note that the equation of state \eqref{EOS} gives the following expression for 
the total momentum 
\begin{equation}\label{momentum.total}
\MM =\rho \vv + \Durg_a\times\Burg^a.
\end{equation}


\section{Nondimensionalization}
\textcolor{blue}{Hyp: 
\begin{itemize}
\item One dimensional along $x\equiv x^{1}$ 
\item $s\equiv 0$
\end{itemize}}

Let group the unknowns in a single vector $q(x,t)$
\begin{equation}
    q(x,t)=\left(\rho,\right)
\end{equation}
\section{Small amplitude waves}

For the plane-wave analysis it is sufficient to consider the nonlinear model~\eqref{PDE} in the 
small deformation limit, i.e. we assume that the distortion field can be decomposed as $ \Dist = 
\bm{I} + \Distsmall $, with $ \bm{I} $ being the identity matrix and $ \Distsmall $ being the 
non-symmetric small strain tensor $ \Distsmall \neq \Distsmall^\transpose $.



\printbibliography

\end{document}
% Acta Mechanica
% Archive of Applied Mechanics
% Mathematics and Mechanics of Solids
% CMAT