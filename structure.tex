%%%%%%%%%%%%%%%%%%%%%%%%%%%%%%%%%%%%%%%%%
% Arsclassica Article
% Structure Specification File
%
% This file has been downloaded from:
% http://www.LaTeXTemplates.com
%
% Original author:
% Lorenzo Pantieri (http://www.lorenzopantieri.net) with extensive modifications by:
% Vel (vel@latextemplates.com)
%
% License:
% CC BY-NC-SA 3.0 (http://creativecommons.org/licenses/by-nc-sa/3.0/)
%
%%%%%%%%%%%%%%%%%%%%%%%%%%%%%%%%%%%%%%%%%

%----------------------------------------------------------------------------------------
%	REQUIRED PACKAGES
%----------------------------------------------------------------------------------------

\usepackage[
nochapters, % Turn off chapters since this is an article        
beramono, % Use the Bera Mono font for monospaced text (\texttt)
eulermath,% Use the Euler font for mathematics
pdfspacing, % Makes use of pdftex’ letter spacing capabilities via the microtype package
dottedtoc % Dotted lines leading to the page numbers in the table of contents
]{classicthesis} % The layout is based on the Classic Thesis style



\usepackage[hmarginratio=1:1,top=25mm,left=20mm,columnsep=25pt]{geometry}
\usepackage{relsize} % e.g. used for \mathsmaller
\usepackage{bm}

\usepackage{arsclassica} % Modifies the Classic Thesis package

\usepackage[T1]{fontenc} % Use 8-bit encoding that has 256 glyphs

\usepackage[utf8]{inputenc} % Required for including letters with accents

\usepackage{graphicx} % Required for including images
\graphicspath{{Figures/}} % Set the default folder for images

\usepackage{enumitem} % Required for manipulating the whitespace between and within lists

\usepackage{lipsum} % Used for inserting dummy 'Lorem ipsum' text into the template

\usepackage{subfig} % Required for creating figures with multiple parts (subfigures)

\usepackage{amsmath,amssymb,amsthm} % For including math equations, theorems, symbols, etc

\usepackage{varioref} % More descriptive referencing

\usepackage{accents}

\usepackage[bottom]{footmisc}

\usepackage{titling} % it needs to define \thanksmarkseries

%----------------------------------------------------------------------------------------
% NEW COMMANDS
%----------------------------------------------------------------------------------------
\renewcommand{\AA}{\bm{A}}
\newcommand{\GG}{\bm{G}}
\newcommand{\MM}{\bm{M}}
\newcommand{\Id}{\bm{I}}
\newcommand{\vv}{\bm{v}}
\newcommand{\QQ}{\bm{Q}}
\renewcommand{\SS}{\bm{S}}
\newcommand{\Lie}{\mathfrak{L}}
\newcommand{\calE}{\mathcal{E}}						%

\newcommand{\IP}[1]{{\color{Red}[IP:\ \ #1]}}
\newcommand{\ER}[1]{{\color{Green}ER:\ \ #1}}
\newcommand{\BL}[1]{{\color{Cerulean}BL:\ \ #1}}
\newcommand{\NF}[1]{{\color{Plum}NF:\ \ #1}}


\newcommand{\sA}{\mathsmaller A}
\newcommand{\sB}{\mathsmaller B}
\newcommand{\sC}{\mathsmaller C}
\newcommand{\sD}{\mathsmaller D}
\newcommand{\sM}{\mathsmaller M}
\newcommand{\sN}{\mathsmaller N}
\newcommand{\sL}{\mathsmaller L}
\newcommand{\kronecker}[2]{\delta^{#1}_{\phantom{#1}#2}}
\newcommand{\durg}[2]{ D_{#1}^{\phantom{#1}#2} }	% Eulerian components of 
%the D field
\newcommand{\LeviCivitaUp}[1]{\varepsilon^{#1}}

\newcommand{\pd}{\partial}
\newcommand{\F}[2]{F^{\ #1}_{\mathsmaller#2}}
\newcommand{\hatF}[2]{\hat{F}^{\ #1}_{\mathsmaller#2}}
\newcommand{\A}[2]{A^{\mathsmaller#1}_{\ #2}}

\newcommand{\dist}[2]{ A^{#1}_{\phantom{#1}#2} }	% Component of the 
\newcommand{\Stress}[2]{ \Sigma^{#1}_{\phantom{#1}#2} }	% Total stress tensor
%distortion field A
\newcommand{\distsmall}[2]{ a_{{#1}{#2}} }	% Small distortion field A 
\newcommand{\Dist}{ \bm{A} }	% Distortion field A in matrix notations
\newcommand{\Burg}{ \bm{B} }	% Burgers tensor = curl(A)
\newcommand{\Durg}{ \bm{D} }	% Complimentary to the Burgers
\newcommand{\Distsmall}{ \bm{a} }	% Small distortion field A in matrix 
%notations
\newcommand{\Plastsmall}{ \bm{\pi} }	% Small distortion field A in matrix 
%notations
\newcommand{\Defgrad}{ \bm{F} }
\newcommand{\iDist}{ \bm{E} }
\newcommand{\symA}{\text{sym}(\bm{a})}
\newcommand{\skewA}{\text{skew}(\bm{a})}
\newcommand{\symP}{\text{sym}(\bm{\Plastsmall})}
\newcommand{\burg}[2]{ B^{{#1}{#2}} }	% Eulerian components of the B field
\newcommand{\itetr}[2]{e^{\phantom{#2}#1}_{#2}}
\newcommand{\tetr}[2]{a^{#1}_{\phantom{#1}#2}}
\newcommand{\rtetr}[2]{a^{#1}_{(\text{r}) #2}}
\newcommand{\spin}[2]{\omega^{#1}_{\phantom{#1}#2}}
\newcommand{\Lor}[2]{\Lambda^{#1'}_{\phantom{#1}#2}}
\newcommand{\iLor}[2]{\Lambda^{\phantom{#2}#1}_{#2'}}
\newcommand{\vel}[1]{v^{#1}}
\newcommand{\D}[1]{\mathcal{D}_{#1}} % Fock-Ivanencko cov derivative
\newcommand{\Tors}[2]{T^{#1}_{\phantom{a}#2}}
\newcommand{\Supp}[2]{S_{#1}^{\phantom{a}#2}}	%supepotential
\newcommand{\Torsl}[1]{T_{#1}}
\newcommand{\ET}[2]{E^{#1}_{\phantom{#1}#2}}	%Torsion decomposition, analog 
%of Electric field
\newcommand{\eT}[2]{D_{#1}^{\phantom{#1}#2}}	%Torsion decomposition, analog 
%of Electric field
\newcommand{\BT}[2]{B^{#1#2}}	%Torsion decomposition, analog of Magnetic field
\newcommand{\hT}[2]{H^{#1#2}}	%Torsion decomposition, analog of Magnetic field
\newcommand{\W}[2]{\mathcal{W}^{#1}_{\phantom{#1}#2}}
\newcommand{\w}[2]{W^{#1}_{\phantom{#1}#2}}
\newcommand{\FI}{Fock-Ivanenko}
\newcommand{\We}{Weitzenb\"ock}
\newcommand{\Lag}{\mathcal{L}}	% Lagrangian which depends on ordinary 
%derivatives
\newcommand{\Lagcov}{\pounds}% Lagrangian which depends on gauge covariant 
%derivatives
\newcommand{\Laghodge}{L}% Lagrangian which depends on the Hodge dual of the 
%torsion
\newcommand{\Lagtors}{\mathbb{L}}% Lagrangian which depends on torsion
\newcommand{\LagBE}{\mathfrak{L}}% Lagrangian which depends on the B and E 
%fields
\newcommand{\veps}{\varepsilon}
\newcommand{\EM}[2]{\Sigma^{#1}_{\phantom{#1}#2}}
\newcommand{\transpose}{{\mathrm {\mathsmaller T}}}
\newcommand{\tr}{\text{tr}}

\newcommand{\tegr}{TEGR}
\newcommand{\HT}[1]{\accentset{\star}{T}^{#1}}

\newcommand{\csh}{c_\text{sh}}	% 	shear sound speed
\newcommand{\csp}{c_\text{sp}}	% 	a sound speed related to the spin
\newcommand{\cinf}{c_\infty}	% 	a sound speed related to 
%1/\sqrt{\mu\eps}
%----------------------------------------------------------------------------------------
%	THEOREM STYLES
%---------------------------------------------------------------------------------------

\theoremstyle{definition} % Define theorem styles here based on the definition style (used for definitions and examples)
\newtheorem{definition}{Definition}

\theoremstyle{plain} % Define theorem styles here based on the plain style (used for theorems, lemmas, propositions)
\newtheorem{theorem}{Theorem}

\theoremstyle{remark} % Define theorem styles here based on the remark style (used for remarks and notes)

%----------------------------------------------------------------------------------------
%	HYPERLINKS
%---------------------------------------------------------------------------------------




%----------------------------------------------------------------------------------------
%	BIBLATEX
%---------------------------------------------------------------------------------------

\usepackage[backend=bibtex,giveninits=true,url=false,doi=true,eprint=true,isbn=false,
backref,backrefstyle=none,maxbibnames=99]{biblatex}
\DefineBibliographyStrings{english}{%
  backrefpage = {Cited on p\adddot},%
  backrefpages = {Cited on pp\adddot}%
}

\bibliography{library}

\renewcommand*{\bibfont}{\footnotesize}

% in order to suppress 'In:'
\renewbibmacro{in:}{%
  \ifboolexpr{%
     test {\ifentrytype{article}}%
  }{}{\printtext{\bibstring{in}\intitlepunct}}%
}

%----------------------------------------------------------------------------------------
% these commands allow to put equations in a fancy boxes:
%----------------------------------------------------------------------------------------
\usepackage{empheq}
\newlength\mytemplen
\newsavebox\mytempbox
\makeatletter
\definecolor{cream}{rgb}{.81, .88, 1}
 \newcommand\mycreambox{%
     \@ifnextchar[%]
        {\@mycreambox}%
        {\@mycreambox[0pt]}}
 \def\@mycreambox[#1]{%
     \@ifnextchar[%]
        {\@@mycreambox[#1]}%
        {\@@mycreambox[#1][0pt]}}
 \def\@@mycreambox[#1][#2]#3{
     \sbox\mytempbox{#3}%
     \mytemplen\ht\mytempbox
     \advance\mytemplen #1\relax
     \ht\mytempbox\mytemplen
     \mytemplen\dp\mytempbox
     \advance\mytemplen #2\relax
     \dp\mytempbox\mytemplen
     \colorbox{cream}{\hspace{1em}\usebox{\mytempbox}\hspace{1em}}}
 \makeatother

% ------------------------------------------------------------------------------
\newcommand*\samethanks[1][\value{footnote}]{\footnotemark[#1]}

